% Minimal Descriptive Programming Language
%
% File:         min.tex
% Author:       Bob Walton (walton@deas.harvard.edu)
% Date:		See \date below.
  
\documentclass[12pt]{article}

\usepackage{makeidx}
\usepackage{pictex}

\makeindex

\setlength{\oddsidemargin}{0in}
\setlength{\evensidemargin}{0in}
\setlength{\textwidth}{6.5in}
\raggedbottom

\setlength{\unitlength}{1in}

\pagestyle{headings}
\setlength{\parindent}{0.0in}
\setlength{\parskip}{1ex}

\setcounter{secnumdepth}{5}
\setcounter{tocdepth}{5}
\newcommand{\subsubsubsection}[1]{\paragraph[#1]{#1.}}
\newcommand{\subsubsubsubsection}[1]{\subparagraph[#1]{#1.}}

% Begin \tableofcontents surgery.

\newcount\AtCatcode
\AtCatcode=\catcode`@
\catcode `@=11	% @ is now a letter

\renewcommand{\contentsname}{}
\renewcommand\l@section{\@dottedtocline{1}{0.1em}{1.4em}}
\renewcommand\l@table{\@dottedtocline{1}{0.1em}{1.4em}}
\renewcommand\tableofcontents{%
    \begin{list}{}%
	     {\setlength{\itemsep}{0in}%
	      \setlength{\topsep}{0in}%
	      \setlength{\parsep}{1ex}%
	      \setlength{\labelwidth}{0in}%
	      \setlength{\baselineskip}{1.5ex}%
	      \setlength{\leftmargin}{1.0in}%
	      \setlength{\rightmargin}{1.0in}}%
    \item\@starttoc{toc}%
    \end{list}}
\renewcommand\listoftables{%
    \begin{list}{}%
	     {\setlength{\itemsep}{0in}%
	      \setlength{\topsep}{0in}%
	      \setlength{\parsep}{1ex}%
	      \setlength{\labelwidth}{0in}%
	      \setlength{\baselineskip}{1.5ex}%
	      \setlength{\leftmargin}{1.0in}%
	      \setlength{\rightmargin}{1.0in}%
	      }%
    \item\@starttoc{lot}%
    \end{list}}

\catcode `@=\AtCatcode	% @ is now restored

% End \tableofcontents surgery.

\newcommand{\CN}[2]%	Change Notice.
    {\hspace*{0in}\marginpar{\sloppy \raggedright \it \footnotesize
     $^{\mbox{#1}}$#2}}
    % Change notice.

\newcommand{\key}[1]{{\em #1}\index{#1}}
\newcommand{\mkey}[2]{{\em #1}\index{#1!#2}}
\newcommand{\skey}[2]{{\em #1#2}\index{#1}}
\newcommand{\ikey}[2]{{\em #1}\index{#2}}
\newcommand{\ttkey}[1]{{\tt #1}\index{#1@{\tt #1}}}
\newcommand{\ttmkey}[2]{{\tt #1}\index{#1@{\tt #1}!#2}}
\newcommand{\ttfkey}[2]{{\tt #1}\index{#1@{\tt #1}!for #2@for {\tt #2}}}
\newcommand{\ttakey}[2]{{\tt #1}\index{#2@{\tt #1}}}
\newcommand{\ttamkey}[3]{{\tt #1}\index{#2@{\tt #1}!#3}}
\newcommand{\ttindex}[1]{\index{#1@{\tt #1}}}
\newcommand{\ttmindex}[2]{\index{#1@{\tt #1}!#2}}
\newcommand{\emkey}[1]{{\em #1}\index{#1@{\em #1}}}
\newcommand{\emindex}[1]{\index{#1@{\em #1}}}

\newcommand{\secref}[1]{\ref{#1}{ p\pageref{#1}}}
\newcommand{\Secref}[1]{\ref{#1}{(p\pageref{#1})}}
\newcommand{\stepref}[1]{\ref{#1}{(p\pageref{#1})}}
\newcommand{\appref}[1]{\ref{#1}{ p\pageref{#1}}}
\newcommand{\pagref}[1]{p\pageref{#1}}

\newcommand{\EOL}{\penalty \exhyphenpenalty}

\newcount\TildeCatcode
\TildeCatcode=\catcode`\~
\catcode`~=12
\newcommand{\Tilde}{~}
\catcode`~=\TildeCatcode

\newcount\CircumflexCatcode
\CircumflexCatcode=\catcode`\^
\catcode`^=12
\newcommand{\Circumflex}{^}
\catcode`^=\CircumflexCatcode

\newcount\CurlyBraCatcode
\newcount\CurlyKetCatcode
\newcount\SquareBraCatcode
\newcount\SquareKetCatcode
\CurlyBraCatcode=\catcode`{
\CurlyKetCatcode=\catcode`}
\SquareBraCatcode=\catcode`[
\SquareKetCatcode=\catcode`]

\catcode`{=\SquareBraCatcode
\catcode`}=\SquareKetCatcode
\catcode`[=\CurlyBraCatcode
\catcode`]=\CurlyKetCatcode

\newcommand[\CurlyBra][{]
\newcommand[\CurlyKet][}]

\catcode`{=\CurlyBraCatcode
\catcode`}=\CurlyKetCatcode
\catcode`[=\SquareBraCatcode
\catcode`]=\SquareKetCatcode

\newcommand{\ttbrackets}{%
    \renewcommand{\{}{\CurlyBra}%
    \renewcommand{\}}{\CurlyKet}}

\newsavebox{\TILDEBOX}
\begin{lrbox}{\TILDEBOX}
\verb|~|
\end{lrbox}
\newcommand{\TILDE}{\usebox{\TILDEBOX}}

\newsavebox{\BACKSLASHBOX}
\begin{lrbox}{\BACKSLASHBOX}
\verb|\|
\end{lrbox}
\newcommand{\BACKSLASH}{\usebox{\BACKSLASHBOX}}

\newsavebox{\LEFTBRACKETBOX}
\begin{lrbox}{\LEFTBRACKETBOX}
\verb|{|
\end{lrbox}
\newcommand{\LEFTBRACKET}{\usebox{\LEFTBRACKETBOX}}

\newsavebox{\RIGHTBRACKETBOX}
\begin{lrbox}{\RIGHTBRACKETBOX}
\verb|}|
\end{lrbox}
\newcommand{\RIGHTBRACKET}{\usebox{\RIGHTBRACKETBOX}}

\newlength{\figurewidth}
\setlength{\figurewidth}{\textwidth}
\addtolength{\figurewidth}{-0.40in}

\newsavebox{\figurebox}

\newenvironment{boxedfigure}[1][!btp]%
	{\begin{figure*}[#1]
	 \begin{lrbox}{\figurebox}
	 \begin{minipage}{\figurewidth}

	 \vspace*{1ex}}%
	{
	 \vspace*{1ex}

	 \end{minipage}
	 \end{lrbox}
	 \begin{center}
	 \fbox{\hspace*{0.1in}\usebox{\figurebox}\hspace*{0.1in}}
	 \end{center}
	 \end{figure*}}

\newenvironment{indpar}[1][0.3in]%
	{\begin{list}{}%
		     {\setlength{\itemsep}{0in}%
		      \setlength{\topsep}{0in}%
		      \setlength{\parsep}{1ex}%
		      \setlength{\labelwidth}{#1}%
		      \setlength{\leftmargin}{#1}%
		      \addtolength{\leftmargin}{\labelsep}}%
	 \item}%
	{\end{list}}

\begin{document}
        
\title{Minimal\\Descriptive and Programming\\Language\\[2ex]MIN\\[2ex]
       (Draft 1a)}

\author{Robert L. Walton\thanks{Copyright 2004 Robert L. Walton.
Permission to copy this document verbatim is granted by the author
to the public.  This document was partly inspired
by my son's efforts at game design.}}

\date{July 30, 2004}
 
\maketitle

\newpage
\begin{center}
\large \bf Table of Contents
\end{center}

\bigskip

\tableofcontents 

\newpage

\section{Introduction}

This document describes MIN, the Minimal Descriptive and Programming
Language.

The main goal of MIN is to make it easy to describe objects,
and to write small pieces of program code change descriptions
in response to external inputs.  MIN has a geometry engine that translates
object descriptions into a geometrical scene, video and audio engines
that translate the geometry and object descriptions into visual and
audio displays, and a view engine that translates object descriptions
into textual displays such as spreadsheets.

In is intended to be the smallest, simplest language that can perform
these tasks well.


\section{Remarks}


\section{Overview}


\section{Lexical Scans}

MIN descriptions are representable by strings of characters that
can be stored in files.  Then the MIN processor reads these, it
scans the characters from left to right to produce a sequence
of \skey{lexeme}s.  There are five kinds of lexemes: words, numbers,
brackets, marks, and quoted strings.

A \key{word} is a string of letters.  E.g., {\tt fie}.

A \key{number} is a string of digits plus an optional \key{decimal point}.
E.g., {\tt 5.2}.

A \key{bracket} is a single bracket character.  E.g., \verb|[| and \verb|]|.

A \key{mark} is a string of \key{mark character}s, which are just characters
that are not letters or digits or decimal points or brackets or quotes.
E.g., \verb|+|.

A \key{quoted string} is a string of characters that begins and ends with
a \verb|"| \key{quote character}.  The string of characters may include
spaces and escape sequences.  See~\secref{QUOTED-STRINGS} for details on
quoted strings.

Words, marks, and quoted strings are all \skey{symbol}s, which are just
character strings that can be used like words or marks.  A quoted string
may be just an alternate representation of a word or a mark.  Thus
{\tt hello} and {\tt "hello"} represent the same thing, the same symbol.
Similarly {\tt :} and {\tt ":"} represent the same symbol.  However
{\tt 0.123} is a number which is not a symbol, and {\tt "0.123"} is a symbol
which is not a number.

There are special rules for some characters.

A \ttkey{_} is a letter.

A \ttkey{.} is a decimal point if followed by a digit, a letter if followed
by another letter, and a mark character otherwise.  E.g., {\tt 0.123} is
a number with a decimal pointer, {\tt .terminator} is a word beginning with
a letter, and {\tt `Hello.'} contains the word {\tt Hello}
followed by the \verb|.| mark.

Users of MIN should not create words that begin with {\tt .}; such words
are reserved for use by the designers of MIN.

A \ttkey{!} is a letter if followed by another letter, and a mark character
otherwise.  E.g., {\tt !set} is a word and {\tt `Hello!'} contains the
word {\tt Hello} followed by the \verb|!| mark.

A \ttkey{\BACKSLASH} is a mark character if it occurs outside quoted
strings, and is an \key{escape character} if it occurs inside quoted strings.
As an escape character it is used to begin a sequence of characters called
an escape sequence that denotes a single character.
See~\secref{ESCAPE-SEQUENCES} for details on escape sequences.

White-space characters, the \key{space character}, the \key{tab character},
the \key{new line character}, the {\tt form feed character}, and the
{\tt vertical tab} character are used to separate lexemes.  The space
character can be used to represent a single space inside quoted strings.
The tab character is always equivalent to one or more space characters,
with tabs set every 8 columns.
The other whitespace characters have special interpretation inside quoted
strings.

\subsection{Quoted Strings}

TBD

\section{Data}

A \key{datum} in MIN is either an atom, an object, or an arrow.

An \key{atom} is either a symbol (word, mark, or quoted string) or a
number.

An object is just a place in memory, and is like a dot on a blank page.
It can be the source or target of arrows, and it is different from
every other object and from every atom.  But it has nothing else.

There are two kinds of arrows: single and double.

A \key{single arrow} is an arrow from an object to either another object
or to an atom.  The arrow has a label, which is a sequence of zero or
more atoms.

A \key{double arrow} is a double headed arrow between two objects.
It has a separate label for each direction, with each label being a
sequence of zero or more atoms.  A double arrow is equivalent to a related pair
of single arrows going in opposite directions between the same two objects.

Two arrows leaving the same object may not have the same label.  Thus
arrows can be named by giving an object which is the source of the arrow
and a label which is the label of a single arrow or the label in the
direction leaving the object of a double arrow.






\bibliographystyle{plain}
\bibliography{min}

\printindex

\end{document}


