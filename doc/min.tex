% Minimal Descriptive Programming Language
%
% File:         min.tex
% Author:       Bob Walton (walton@deas.harvard.edu)
% Date:        	See \date below.
  
\documentclass[12pt]{article}

\usepackage{makeidx}
\usepackage{pictex}

\makeindex

\setlength{\oddsidemargin}{0in}
\setlength{\evensidemargin}{0in}
\setlength{\textwidth}{6.5in}
\raggedbottom

\setlength{\unitlength}{1in}

\pagestyle{headings}
\setlength{\parindent}{0.0in}
\setlength{\parskip}{1ex}

\setcounter{secnumdepth}{5}
\setcounter{tocdepth}{5}
\newcommand{\subsubsubsection}[1]{\paragraph[#1]{#1.}}
\newcommand{\subsubsubsubsection}[1]{\subparagraph[#1]{#1.}}

% Begin \tableofcontents surgery.

\newcount\AtCatcode
\AtCatcode=\catcode`@
\catcode `@=11	% @ is now a letter

\renewcommand{\contentsname}{}
\renewcommand\l@section{\@dottedtocline{1}{0.1em}{1.4em}}
\renewcommand\l@table{\@dottedtocline{1}{0.1em}{1.4em}}
\renewcommand\tableofcontents{%
    \begin{list}{}%
	     {\setlength{\itemsep}{0in}%
	      \setlength{\topsep}{0in}%
	      \setlength{\parsep}{1ex}%
	      \setlength{\labelwidth}{0in}%
	      \setlength{\baselineskip}{1.5ex}%
	      \setlength{\leftmargin}{1.0in}%
	      \setlength{\rightmargin}{1.0in}}%
    \item\@starttoc{toc}%
    \end{list}}
\renewcommand\listoftables{%
    \begin{list}{}%
	     {\setlength{\itemsep}{0in}%
	      \setlength{\topsep}{0in}%
	      \setlength{\parsep}{1ex}%
	      \setlength{\labelwidth}{0in}%
	      \setlength{\baselineskip}{1.5ex}%
	      \setlength{\leftmargin}{1.0in}%
	      \setlength{\rightmargin}{1.0in}%
	      }%
    \item\@starttoc{lot}%
    \end{list}}

\catcode `@=\AtCatcode	% @ is now restored

% End \tableofcontents surgery.

\newcommand{\CN}[2]%	Change Notice.
    {\hspace*{0in}\marginpar{\sloppy \raggedright \it \footnotesize
     $^{\mbox{#1}}$#2}}
    % Change notice.

\newcommand{\key}[1]{{\bf \em #1}\index{#1}}
\newcommand{\mkey}[2]{{\bf \em #1}\index{#1!#2}}
\newcommand{\skey}[2]{{\bf \em #1#2}\index{#1}}
\newcommand{\ikey}[2]{{\bf \em #1}\index{#2}}
\newcommand{\ttkey}[1]{{\tt \bf #1}\index{#1@{\tt #1}}}
% < and > do not work for \tt \bf, hence:
\newcommand{\ttnbkey}[1]{{\tt #1}\index{#1@{\tt #1}}}
\newcommand{\ttmkey}[2]{{\tt \bf #1}\index{#1@{\tt #1}!#2}}
\newcommand{\ttmnbkey}[2]{{\tt #1}\index{#1@{\tt #1}!#2}}
\newcommand{\ttfkey}[2]{{\tt \bf #1}\index{#1@{\tt #1}!for #2@for {\tt #2}}}
\newcommand{\ttakey}[2]{{\tt \bf #1}\index{#2@{\tt #1}}}
\newcommand{\ttamkey}[3]{{\tt \bf #1}\index{#2@{\tt #1}!#3}}
\newcommand{\ttindex}[1]{\index{#1@{\tt #1}}}
\newcommand{\ttmindex}[2]{\index{#1@{\tt #1}!#2}}
\newcommand{\emkey}[1]{{\bf \em #1}\index{#1@{\em #1}}}
\newcommand{\emindex}[1]{\index{#1@{\em #1}}}

\newcommand{\secref}[1]{\ref{#1}$^{p\pageref{#1}}$}
\newcommand{\stepref}[1]{\ref{#1}$^{p\pageref{#1}}$}
\newcommand{\appref}[1]{\ref{#1}$^{p\pageref{#1}}$}
\newcommand{\figref}[1]{\ref{#1}$^{p\pageref{#1}}$}
\newcommand{\pagref}[1]{p\pageref{#1}}

\newcommand{\EOL}{\penalty \exhyphenpenalty}

\newcount\TildeCatcode
\TildeCatcode=\catcode`\~
\catcode`~=12
\newcommand{\Tilde}{~}
\catcode`~=\TildeCatcode

\newcount\CircumflexCatcode
\CircumflexCatcode=\catcode`\^
\catcode`^=12
\newcommand{\Circumflex}{^}
\catcode`^=\CircumflexCatcode

\newcount\CurlyBraCatcode
\newcount\CurlyKetCatcode
\newcount\SquareBraCatcode
\newcount\SquareKetCatcode
\CurlyBraCatcode=\catcode`{
\CurlyKetCatcode=\catcode`}
\SquareBraCatcode=\catcode`[
\SquareKetCatcode=\catcode`]

\catcode`{=\SquareBraCatcode
\catcode`}=\SquareKetCatcode
\catcode`[=\CurlyBraCatcode
\catcode`]=\CurlyKetCatcode

\newcommand[\CurlyBra][{]
\newcommand[\CurlyKet][}]

\catcode`{=\CurlyBraCatcode
\catcode`}=\CurlyKetCatcode
\catcode`[=\SquareBraCatcode
\catcode`]=\SquareKetCatcode

\newcommand{\ttbrackets}{%
    \renewcommand{\{}{\CurlyBra}%
    \renewcommand{\}}{\CurlyKet}}

\newsavebox{\TILDEBOX}
\begin{lrbox}{\TILDEBOX}
\verb|~|
\end{lrbox}
\newcommand{\TILDE}{\usebox{\TILDEBOX}}

\newsavebox{\BACKSLASHBOX}
\begin{lrbox}{\BACKSLASHBOX}
\verb|\|
\end{lrbox}
\newcommand{\BACKSLASH}{\usebox{\BACKSLASHBOX}}

\newsavebox{\LEFTBRACKETBOX}
\begin{lrbox}{\LEFTBRACKETBOX}
\verb|{|
\end{lrbox}
\newcommand{\LEFTBRACKET}{\usebox{\LEFTBRACKETBOX}}

\newsavebox{\RIGHTBRACKETBOX}
\begin{lrbox}{\RIGHTBRACKETBOX}
\verb|}|
\end{lrbox}
\newcommand{\RIGHTBRACKET}{\usebox{\RIGHTBRACKETBOX}}

\newsavebox{\UNDERLINEBOX}
\begin{lrbox}{\UNDERLINEBOX}
\verb|_|
\end{lrbox}
\newcommand{\UNDERLINE}{\usebox{\UNDERLINEBOX}}

\newsavebox{\CIRCUMFLEXBOX}
\begin{lrbox}{\CIRCUMFLEXBOX}
\verb|^|
\end{lrbox}
\newcommand{\CIRCUMFLEX}{\usebox{\CIRCUMFLEXBOX}}

\newsavebox{\BARBOX}
\begin{lrbox}{\BARBOX}
\verb/|/
\end{lrbox}
\newcommand{\BAR}{\usebox{\BARBOX}}

\newsavebox{\LESSTHANBOX}
\begin{lrbox}{\LESSTHANBOX}
\verb/</
\end{lrbox}
\newcommand{\LESSTHAN}{\usebox{\LESSTHANBOX}}

\newsavebox{\GREATERTHANBOX}
\begin{lrbox}{\GREATERTHANBOX}
\verb/>/
\end{lrbox}
\newcommand{\GREATERTHAN}{\usebox{\GREATERTHANBOX}}

\newlength{\figurewidth}
\setlength{\figurewidth}{\textwidth}
\addtolength{\figurewidth}{-0.40in}

\newsavebox{\figurebox}

\newenvironment{boxedfigure}[1][!btp]%
	{\begin{figure*}[#1]
	 \begin{lrbox}{\figurebox}
	 \begin{minipage}{\figurewidth}

	 \vspace*{1ex}}%
	{
	 \vspace*{1ex}

	 \end{minipage}
	 \end{lrbox}
	 \begin{center}
	 \fbox{\hspace*{0.1in}\usebox{\figurebox}\hspace*{0.1in}}
	 \end{center}
	 \end{figure*}}

\newenvironment{indpar}[1][0.3in]%
	{\begin{list}{}%
		     {\setlength{\itemsep}{0in}%
		      \setlength{\topsep}{0in}%
		      \setlength{\parsep}{1ex}%
		      \setlength{\labelwidth}{#1}%
		      \setlength{\leftmargin}{#1}%
		      \addtolength{\leftmargin}{\labelsep}}%
	 \item}%
	{\end{list}}

\begin{document}
        
\title{Minimal\\Descriptive Programming\\Language\\[2ex]MIN\\[2ex]
       (Draft 1a)}

\author{Robert L. Walton\thanks{Copyright 2005 Robert L. Walton.
Permission to copy this document verbatim is granted by the author
to the public.  This document was partly inspired
by my son's budding career as a game designer.}}

\date{May 23, 2005}
 
\maketitle

\newpage
\begin{center}
\large \bf Table of Contents
\end{center}

\bigskip

\tableofcontents 

\newpage

\section{Introduction}

This document describes MIN, the Minimal Descriptive Programming
Language.

The main goal of MIN is to make it easy to describe objects,
and to write small pieces of program code that change object descriptions
in response to external inputs.  MIN has a geometry engine that translates
object descriptions into a geometrical data, video and audio engines
that translate the geometry and object data into visual and
audio displays, and a view engine that translates between object descriptions
and textual displays such as spreadsheets.

MIN is intended to be the smallest, simplest language that can perform
these tasks well.


\section{Remarks}

TBD

\section{Overview}

TBD

\section{Lexemes}
\label{LEXEMES}

MIN descriptions are representable by strings of characters that
can be stored in files.  When a MIN program reads these, it
scans the characters from left to right to produce a sequence
of \skey{lexeme}s.  There are five kinds of lexemes: words, numbers,
brackets, marks, and quoted strings.

A \key{word} is a string of letters.  E.g., {\tt fie}.  The underline
character (\ttmkey{\UNDERLINE}{letter}) is a letter.
A period (\ttmkey{.}{letter})
is a letter if it is followed by a letter; similarly an exclamation
mark (\ttmkey{!}{letter}) is a letter if it is followed by a letter,
and an apostrophe (\ttmkey{'}{letter}) is a letter if it
is followed by a letter.
E.g., \verb|.separator|, \verb|!set|, and \verb|it's| are words.

A \key{number} is a string of digits plus an optional \key{decimal point}.
E.g., {\tt 5.2}.  Any decimal point must be followed by a digit.

A \key{bracket} is a single bracket character.  E.g., \verb|[| and \verb|]|.
The \key{bracket characters} are parentheses,
\ttmkey{(}{bracket character} and \ttmkey{)}{bracket character},
square brackets, \ttmkey{[}{bracket character} and
\ttmkey{]}{bracket character},
curly brackets, \ttmkey{\LEFTBRACKET}{bracket character} and
\ttmkey{\RIGHTBRACKET}{bracket character},
and single quotes, \ttmkey{`}{bracket character} and
\ttmkey{'}{bracket character}.  The close quote (\verb|'|) is
a letter, and not a bracket, if it is followed by a letter.

A \key{mark} is a string of mark characters, which are just characters
that are not letters or digits or decimal points or brackets or quotes
or white-space characters.
E.g., \verb|+| and \verb|::| are marks.  The \key{mark characters} are
comma (\ttmkey{,}{mark character}),
semi-colon (\ttmkey{;}{mark character}),
colon (\ttmkey{:}{mark character}),
period (\ttmkey{.}{mark character}) when not followed by a digit or
letter,
exclamation mark (\ttmkey{!}{mark character}) when not followed by a
letter,
at sign ({\tt @}\index{"@@{\tt "@}!mark character}),
pound sign (\ttmkey{\#}{mark character}),
dollar sign (\ttmkey{\$}{mark character}),
percent sign (\ttmkey{\%}{mark character}),
circumflex (\ttmkey{\CIRCUMFLEX}{mark character}),
ampersand (\ttmkey{\&}{mark character}),
asterisk (\ttmkey{*}{mark character}),
minus sign (\ttmkey{-}{mark character}),
plus sign (\ttmkey{+}{mark character}),
equal sign (\ttmkey{=}{mark character}),
vertical bar (\ttmkey{\BAR}{mark character}),
back-slash (\ttmkey{\BACKSLASH}{mark character}),
less-than (\ttmkey{\LESSTHAN}{mark character}),
greater-than (\ttmkey{\GREATERTHAN}{mark character}),
slash (\ttmkey{/}{mark character}),
and
tilde (\ttmkey{\TILDE}{mark character}).


A \key{quoted string} is a string of characters that begins and ends with
a \verb|"| \key{quote character}.  The string of characters may include
spaces and escape sequences.  See~\secref{QUOTED-STRINGS} for details on
quoted strings.

Words, marks, and quoted strings are all \skey{symbol}s, which are just
character strings that can be used like words or marks.  A quoted string
may be just an alternate representation of a word or a mark.  Thus
{\tt hello} and {\tt "hello"} represent the same thing, the same symbol.
Similarly {\tt :} and {\tt ":"} represent the same symbol.  However
{\tt 0.123} is a number which is not a symbol, and {\tt "0.123"} is a symbol
which is not a number.

There are special rules for some characters.

An underline (\ttmkey{\UNDERLINE}{letter}) is a letter.

A \ttmkey{.}{letter}\index{.@{\tt .}!decimal point}
is a decimal point if followed by a digit, a letter if followed
by another letter, and a mark character otherwise.  E.g., {\tt 0.123} is
a number with a decimal point, {\tt .terminator} is a word beginning with
a letter, and {\tt `Hello.'} contains the word {\tt Hello}
followed by the \verb|.| mark.

Users of MIN should not create words that begin with
\ttmkey{.}{reserved use}; such words
are reserved for use by the designers of MIN.

An exclamation mark ({\tt !}\index{"!@{\tt "!}!letter}%
\index{"!@{\tt "!}!mark character})
is a letter if followed by another letter, and a mark character
otherwise.  E.g., {\tt !set} is a word and {\tt `Hello!'} contains the
word {\tt Hello} followed by the \verb|!| mark.

Similarly the close quote (\ttmkey{'}{letter}) is a letter if followed
by a letter, as in \verb|it's|, and a closing bracket otherwise.

The double quote ({\tt "}\index{""@{\tt ""}}) is used to begin and end
quoted strings.

A \ttkey{\BACKSLASH} is a mark character if it occurs outside quoted
strings, and is an \key{escape character} if it occurs inside quoted strings.
As an escape character it is used to begin a sequence of characters called
an escape sequence that denotes a single character.
See~\pagref{ESCAPE-SEQUENCES} for details on escape sequences.

\ikey{White-space characters}{white-space character} --
the \key{space character}, the \key{tab character},
the \key{new line character}, the \key{form feed character}, and the
\key{vertical tab} character -- are used to separate lexemes.  The space
character can be used to represent a single space inside quoted strings.
The tab character is always equivalent to one or more space characters,
with tabs set every 8 columns.
The other white-space characters have special interpretation inside quoted
strings, see~\secref{QUOTED-STRINGS}.

\subsection{Quoted Strings}
\label{QUOTED-STRINGS}

A \key{quoted string} is an alternative representation of a symbol, the
other representations being words and marks.  The quoted string \verb|"::"|
and the mark \verb|::| represent the same thing.  Quoted strings can
represent symbols that have characters in them which cannot be in words
or marks.

There is one difference between a quoted string and its corresponding
word or mark, e.g.~between \verb|"+"| and \verb|+|.  This is that the
quoted string cannot be interpreted as an operator during parsing, while
the word or mark, if it is an operator, can be so interpreted.
See \secref{PARSER-EXPANSION}.

A quoted string consists of a sequence of
\skey{character representative}s
surrounded by double quotes (\verb|"|\index{""@{\tt ""}}).  A non-white-space
character other than quote (\verb|"|) or backslash (\verb|\|)
can be used to represent itself.  The single space character can be used
to represent itself.  Line feeds represent themselves;
carriage returns are ignored.
The other character representatives are called
\skey{escape sequence}s,\label{ESCAPE-SEQUENCES}
and consist of a backslash (\verb|\|\index{\\@{\tt \BACKSLASH}})
followed by other characters.  Each escape sequence represents exactly
one character, except for the \verb|\+|{\em new-line}~{\em white-space}
sequence that represents zero characters.  The possible escape sequences and
the character they represent are:

\begin{center}
\begin{tabular}{lp{4in}}
\verb|\n| & new line \\
\verb|\r| & carriage return \\
\verb|\t| & horizontal tab \\
\verb|\b| & backspace \\
\verb|\f| & form feed \\
\verb|\v| & vertical tab \\
\verb|\\| & \verb|\| \\
\verb|\"| & \verb|"| \\[1ex]
\verb|\x|$hh$ & character with ASCII code $hh$ as a 2 digit
		hexadecimal number \\[1ex]
\verb|\|$ooo$ & character with ASCII code $ooo$ as a
		3 digit octal number \\[1ex]
\verb|\|{\em new-line} {\em white-space}
		& the backslash, new line, and all following white space
		  characters are replaced by a space character \\[1ex]
\verb|\+|{\em new-line} {\em white-space}
		& the backslash, {\tt +}, new line, and all following
		  white space characters are deleted \\
\end{tabular}
\end{center}%
\index{n@{\tt \BACKSLASH n}}%
\index{r@{\tt \BACKSLASH r}}%
\index{t@{\tt \BACKSLASH t}}%
\index{b@{\tt \BACKSLASH b}}%
\index{f@{\tt \BACKSLASH f}}%
\index{v@{\tt \BACKSLASH v}}%
\index{\\2@{\BACKSLASH\BACKSLASH}}%
\index{""2@{\tt \BACKSLASH""}}%
\index{space character 2@{\BACKSLASH{\em space}}}%
\index{x@{\tt \BACKSLASH x}$hh$}%
\index{digit@{\BACKSLASH{\em digit}}}%
\index{new line 2@{\BACKSLASH{\em new-line} {\em white-space}}}
\index{+@{\BACKSLASH{\tt +} {\em new-line} {\em white-space}}}

Single space characters after a newline in a quoted string
are removed up through the column containing the initial quote (\verb|"|)
of the quoted string.  Thus

\begin{indpar}\begin{verbatim}
some text "This is a line.
           And another line."
\end{verbatim}\end{indpar}

contains a quoted string that represents two lines, the first beginning
with `\verb|This|' and second beginning with `\verb|And|'.  There is no
white-space at the beginning of the second represented line, as the white-space
at the beginning of its representing quoted string line is removed.
The last removed white-space column is that of the \verb|"| beginning the
quoted string.

Tab characters in are always equivalent to one or more space characters,
with tabs set every 8 columns.  Columns are counted with respect to the
beginning of the line, and \underline{not} the beginning of any quoted string.

Form feeds and vertical tabs cannot appear inside a quoted string.  Carriage
returns cannot appear inside or outside a quoted string
except in a sequence of carriage returns and line feeds
containing at least one line feed.
Non-white-space control characters cannot appear inside or outside a quoted
string.  All these characters can be represented by escape sequences inside
a quoted string.



\section{Data}

A \key{datum} in MIN is either an atom, an object, or an arrow.

An \key{atom} is either a symbol (word, mark, or quoted string) or a
number.

An object is just a place in memory, and is like a dot on a blank page.
It can be the source or target of arrows, and it is different from
every other object and from every atom.  But it is nothing more.

However, as a place in memory, an object has a name.  Objects are assigned
\skey{raw object name}s
of the form `\ttmnbkey{\#\#}{in raw object name}$I$' where $I$
is an integer.  Raw object names of this form are assigned only to objects
that must be named in an output stream (e.g., printed output).
The first object named in output stream is assigned the name
\verb|##1|, the second object named in output the name \verb|##2|,
and so forth.  The same object may be assigned different names in different
output streams.\footnote{A possible implementation is to give objects that
have been assigned names in an output stream a hidden system defined output
stream specific attribute (\pagref{ATTRIBUTE})
equal to the object's name integer.}

There are two kinds of arrows: single and double.

A \key{single arrow} is an arrow from an object to either another object
or to an atom.  The arrow has a label, which is a sequence of zero or
more atoms.

A \key{double arrow} is a double headed arrow between two objects.
It has a separate label for each direction, with each label being a
sequence of zero or more atoms.  A double arrow is equivalent to a
\underline{related pair}
of single arrows going in opposite directions between the same two objects.
The difference between a double arrow and a pair of single arrows is that
it is possible to delete only one direction of a pair of single arrows, but 
when deleting a double arrow, both directions are deleted.

An \key{arrow label} is a sequence of zero or more atoms.
Two arrows leaving the same object may have the same label.  Thus
an object and an arrow label together name
a set of arrows sourced at the object.

\ikey{Arrow flags}{arrow flag} may be attached to arrow labels.
More precisely, a set of arrow flags is defined for each object
and each arrow label, and these flags apply to all arrows sourced at the
object that have the given label.
The standard flags are the \key{dot flag} (\ttmkey{.}{dot flag}),
and the \key{maybe flag} (\ttmkey{?}{maybe flag}).
Arrows with a dot flagged label are not to be output when their source
is output.
Targets of arrows with a maybe flagged label may be garbage collected
(made to disappear automatically, see \secref{GARBAGE-COLLECTION})
if they cannot be reached except by traversing arrows whose labels
have maybe flags.

We will give examples in the next section along with a basic way of
representing sets of objects in text.  In the rest of this document
arrows are called \skey{attribute}s,\label{ATTRIBUTE} arrow labels are called
\skey{attribute name}s, arrow flags are called \skey{attribute flag}s,
and arrow targets are called \skey{attribute value}s.
Also `\ikey{attribute L of object O}{attribute!of an object}'
denotes the set of all values (arrow targets) of attributes
of object O (arrows sourced at O) which have the attribute label
(arrow label) L.


\subsection{Raw Representations}

A set of objects can be written to a text file or read from a text file.
When this is done, a textual representation of the object set is used
in the file.  The simplest representation is the raw representation,
which we now describe.

The format of a \key{raw representation} is:

\begin{indpar}
\begin{tabular}{l}
\emkey{raw-representation} ::= {\em raw-object-representation} \ldots
\end{tabular}

\begin{tabular}{l}
\emkey{raw-object-representation} ::=
    {\em raw-object-header}
    {\em raw-attribute-representation} \ldots
\end{tabular}

\begin{tabular}{l}
\emkey{raw-object-header}
    \begin{tabular}[t]{@{}rl@{}}
    ::= & {\em raw-object-name} \ttmkey{::}{in {\em raw-object-header}} \\
    $|$ & {\em raw-object-name} \verb|>|{\tt \bf :}%
          \ttmindex{>:}{in {\em raw-object-header}} \\
    $|$ & {\em raw-object-name} \verb|>>|{\tt \bf :}%
          \ttmindex{>>:}{in {\em raw-object-header}} \\
    \end{tabular}
\end{tabular}

\begin{tabular}{l}
\emkey{raw-attribute-representation}\label{RAW-ATTRIBUTE-REPRESENTATION}
    \begin{tabular}[t]{@{}rl@{}}
    ::= & {\em raw-single-attribute-representation} \\
    $|$ & {\em raw-double-attribute-representation}
    \end{tabular}
\end{tabular}

\begin{tabular}{l}
\emkey{raw-single-attribute-representation} ::=
\\\hspace*{1in}
	{\em attribute-label} {\em attribute-label-terminator}
	{\em single-attribute-value}
	\ttmkey{;;}{in {\em raw-single-attribute}!{\em -representation}}
\end{tabular}%
\index{value!attribute}

\begin{tabular}{l}
\emkey{single-attribute-value} ::= {\em atom} $|$ {\em raw-object-name}
\end{tabular}

\begin{tabular}{l}
\emkey{raw-double-attribute-representation} ::=
\\\hspace*{1in}
	{\em attribute-label} {\em attribute-label-terminator}
	{\em double-attribute-value}
\\\hspace*{1in}
	{\em attribute-label-initiator} {\em attribute-label}
	\ttmkey{;;}{in {\em raw-double-attribute-representation}}
\end{tabular}

\begin{tabular}{l}
\emkey{double-attribute-value} ::= {\em raw-object-name}
\end{tabular}

\begin{tabular}{l}
\emkey{raw-object-name} ::=
	\ttmnbkey{\#\#}{in {\em raw-object-name}} {\em non-negative-integer}
\end{tabular}

\begin{tabular}{l}
\emkey{attribute-label} ::= {\em atom} \ldots
\end{tabular}%
\index{label!attribute}

\begin{tabular}{l}
\emkey{attribute-label-terminator} ::=
	{\em flag-character}\ldots{\tt \bf :}%
	\index{:flag-character@{\em flag-character}\ldots{\tt :}!
	attribute-label-terminator}~~~~~(a mark)
\end{tabular}

\begin{tabular}{l}
\emkey{attribute-label-initiator} ::=
	{\tt \bf :}{\em flag-character}\ldots%
	\index{:flag-character@{\tt :}{\em flag-character}\ldots!
	attribute-label-initiator}~~~~~(a mark)
\end{tabular}

\begin{tabular}{l}
\emkey{flag-character} ::= {\em mark-character}
		except \verb|:|, \verb|<|, or \verb|>|
\end{tabular}

\begin{tabular}{l}
\emkey{atom} ::= {\em non-special-lexeme}
\end{tabular}

\begin{tabular}{l}
\emkey{special-lexeme}
    \begin{tabular}[t]{@{}rl@{}}
    ::= & \verb|##| $|$ \verb|::| $|$ \verb|>:| $|$ \verb|>>:| $|$ \verb|;;| \\
    $|$ & {\em attribute-label-terminator} $|$ {\em attribute-label-initiator}
    \end{tabular}
\end{tabular}
\end{indpar}

A simple example of the raw representation of a set of objects is:

\begin{center}
\begin{tabular}[b]{@{}l@{}}
\verb|##1::|\\
\verb|    type: woman;;|\\
\verb|    name: Jill;;|\\
\verb|    husband: ##2 :wife;;|\\
\verb|##2::|\\
\verb|    type: man;;|\\
\verb|    name: Jack;;|\\
\end{tabular}
~~~~~~~~~
\begin{picture}(3.0,1.5)
\put(0,0){\framebox(3.0,1.5){}}
\put(0.3,1.00){\makebox(0.8,0.3){\tt \#\#1}}
\put(0.7,1.15){\oval(0.8,0.3)}
\put(0.5,1.00){\vector(0,-1){0.5}}
\put(0.45,0.75){\makebox(0,0)[r]{\tt type}}
\put(0.5,0.4){\makebox(0,0){\tt woman}}
\put(0.9,1.00){\vector(0,-1){0.7}}
\put(0.95,0.55){\makebox(0,0)[l]{\tt name}}
\put(0.9,0.2){\makebox(0,0){\tt Jill}}
\put(1.9,1.00){\makebox(0.8,0.3){\tt \#\#2}}
\put(2.3,1.15){\oval(0.8,0.3)}
\put(2.1,1.00){\vector(0,-1){0.5}}
\put(2.05,0.75){\makebox(0,0)[r]{\tt type}}
\put(2.1,0.4){\makebox(0,0){\tt man}}
\put(2.5,1.00){\vector(0,-1){0.7}}
\put(2.55,0.55){\makebox(0,0)[l]{\tt name}}
\put(2.5,0.2){\makebox(0,0){\tt Jack}}
\put(1.1,1.17){\vector(1,0){0.8}}
\put(1.9,1.13){\vector(-1,0){0.8}}
\put(1.5,1.15){\oval(0.07,0.12)}
\put(1.5,1.30){\makebox(0,0){\tt husband}}
\put(1.5,1.00){\makebox(0,0){\tt wife}}
\end{picture}
\end{center}

This represents two objects.
There are two single attributes of object \verb|##1| (arrows sourced
at \verb|##1|),
one attribute labeled {\tt type} whose value (target) is the atom {\tt woman},
and one attribute labeled {\tt name} whose value is the atom {\tt Jill}.
There are two similar single attributes from object \verb|##2|.
There is a double attribute (double arrow)
between the two objects which has the
label {\tt husband} when going from \verb|##1| to \verb|##2|
and the label {\tt wife} when going in the reverse direction.

In order to permit \verb|:| to be used in an attribute label, the following
\key{quote representation rule} is applied to representations.
A quoted symbol cannot be recognized as an atom that has special
meaning in a representation, such as the \verb|::|, \verb|:|,
\verb|##|, or \verb|;;| atoms in raw representations.  Thus if any
of these atoms are to be part of an attribute label or value, they should
be represented by \verb|"::"|, \verb|":"|, \verb|"##"|, or \verb|";;"|.
Other than this rule, there is no
distinction between quoted and unquoted representations of unquoted atoms.

The double-semi-colons (\verb|;;|) at the ends of attribute representations
may be omitted according to the \key{double-semi-colon representation rule}.
This rule says that the \ttmkey{;;}{omitting} at the end of an attribute
representation may be omitted provided the next non-blank line is not
indented with respect to the first non-white-space character of the
attribute representation with the omitted \verb|;;|,
or alternatively if there are no following non-blank lines in the text
(e.g., in the file).
Thus the \verb|;;|'s in the example just given may be omitted.
To detect errors, if a line of an attribute representation is indented by
just one column with respect to the first non-white-space character of the
attribute representation, the line is considered to be in error.
Thus indentations must be by at least two columns.

It is possible to place \skey{attribute flag}s on attribute labels
by putting flag characters
before or after the \verb|:| that follows or precedes an attribute label.
The following is the same as the above example except that flags
have been added to some of the attributes:

\begin{center}
\begin{tabular}[b]{@{}l@{}}
\verb|##1::|\\
\verb|    type: woman;;|\\
\verb|    name-: Jill;;|\\
\verb|    husband@: ##2 :@wife;;|\\
\verb|##2::|\\
\verb|    type: man;;|\\
\verb|    name+: Jack;;|\\
\end{tabular}
~~~~~~~~~
\begin{picture}(3.0,1.5)
\put(0,0){\framebox(3.0,1.5){}}
\put(0.3,1.00){\makebox(0.8,0.3){\tt \#\#1}}
\put(0.7,1.15){\oval(0.8,0.3)}
\put(0.5,1.00){\vector(0,-1){0.5}}
\put(0.45,0.75){\makebox(0,0)[r]{\tt type}}
\put(0.5,0.4){\makebox(0,0){\tt woman}}
\put(0.9,1.00){\vector(0,-1){0.7}}
\put(0.95,0.55){\makebox(0,0)[l]{{\tt name}$^{\mbox{\tt -}}$}}
\put(0.9,0.2){\makebox(0,0){\tt Jill}}
\put(1.9,1.00){\makebox(0.8,0.3){\tt \#\#2}}
\put(2.3,1.15){\oval(0.8,0.3)}
\put(2.1,1.00){\vector(0,-1){0.5}}
\put(2.05,0.75){\makebox(0,0)[r]{\tt type}}
\put(2.1,0.4){\makebox(0,0){\tt man}}
\put(2.5,1.00){\vector(0,-1){0.7}}
\put(2.55,0.55){\makebox(0,0)[l]{{\tt name}$^{\mbox{\tt +}}$}}
\put(2.5,0.2){\makebox(0,0){\tt Jack}}
\put(1.1,1.17){\vector(1,0){0.8}}
\put(1.9,1.13){\vector(-1,0){0.8}}
\put(1.5,1.15){\oval(0.07,0.12)}
\put(1.5,1.30){\makebox(0,0){{\tt husband}$^{\mbox{\tt @}}$}}
\put(1.5,1.00){\makebox(0,0){{\tt wife}$^{\mbox{\tt @}}$}}
\end{picture}
\end{center}

In the picture the attribute flags have been added as superscripts on the
attribute labels, and in the text the flags have been added before of after the
\verb|:| that follows or precedes the attribute label.

Several attributes of the same object (arrows sourced at the object)
may have the same attribute
label.  An example of this, in which object \verb|##1| has two
attributes labeled \verb|child|, is:

\begin{center}
\begin{tabular}[b]{@{}l@{}}
\verb|##1::|\\
\verb|    child: ##2 :parent;;|\\
\verb|    child: ##3 :parent;;|\\
\end{tabular}
~~~~~~~~~
\begin{picture}(3.0,1.5)
\put(0,0){\framebox(3.0,1.5){}}
\put(0.3,1.00){\makebox(2.4,0.3){\tt \#\#1}}
\put(1.5,1.15){\oval(2.4,0.3)}
\put(0.65,1.00){\vector(0,-1){0.6}}
\put(0.60,0.80){\makebox(0,0)[r]{\tt child}}
\put(0.75,0.40){\vector(0,1){0.6}}
\put(0.80,0.55){\makebox(0,0)[l]{\tt parent}}
\put(2.25,1.00){\vector(0,-1){0.6}}
\put(2.20,0.80){\makebox(0,0)[r]{\tt child}}
\put(2.35,0.40){\vector(0,1){0.6}}
\put(2.40,0.55){\makebox(0,0)[l]{\tt parent}}
\put(0.3,0.10){\makebox(0.8,0.3){\tt \#\#2}}
\put(0.7,0.25){\oval(0.8,0.3)}
\put(1.9,0.10){\makebox(0.8,0.3){\tt \#\#3}}
\put(2.3,0.25){\oval(0.8,0.3)}
\end{picture}
\end{center}

We say that the value of the \verb|child| attribute of \verb|##1| is
the set to two elements, \verb|##2| and \verb|##3|.

Differences between \ttnbkey{::}, \ttnbkey{>:}, are \ttnbkey{>>:} relate
to what is done when an object or attribute label previously exists.

\verb|::| indicates that the object being represented should not previously
exist, or if it does exist, must not have been defined by any previous
object representation (it may have been defined as the value of a double
attribute).  Second, any attribute label represented in the object
representation, if it previously exists, must be represented with exactly
the same flags as it already has.

At the other extreme, \verb|>>:| adds to existing objects.
The object being represented can previously exist.  Any attribute
representation in the object representation creates a new attribute value.
Any flags on an attribute label are added to the flags of the label if
the label already exists.

\verb|>:| is like \verb|>>:| except that the object representation cannot
add new values to previously existing attributes of the object represented.
Thus \verb|>:| is used to introduce new attributes to an existing object.
Note that new values can be added to attributes of double attribute value
objects.

Note that double attributes must have only one representation.  If they
are given two representations, one for each end of the attribute (arrow
pair), \underline{two} identical double attributes (two arrow pairs with the
same labels) will be created.  Usually one end of a double attribute is
thought of as the primary end, and its object representation is used to
include the sole representation of the double attribute.

Object \ttnbkey{\#\#0} is special: it is the \ttkey{.GLOBAL} object, and
its attributes are called \skey{global variables}.  One of the global
variables is named {\tt .GLOBAL} and has as its value the {\tt .GLOBAL}
object, a situation which can be achieved by the data representation:



\begin{indpar}\begin{verbatim}
##0>:
    .GLOBAL: ##0
\end{verbatim}\end{indpar}



\subsection{Cooked Representations}

In contrast to the raw representation of a set of objects there is the
cooked representation, that is much easier to read and write, but more
long winded to explain.

The main thing that the cooked representation does is take certain objects
that are organized like lists and represent them as lists.

For example,

\begin{indpar}\begin{verbatim}
##93:: This is a sentence.
\end{verbatim}\end{indpar}

is the cooked representation of the object

\begin{indpar}\begin{verbatim}
##93::
    1: this
    2: is
    3: a
    4: sentence
    .terminator: "."
    .initiator: capital
\end{verbatim}\end{indpar}

The following is a second example in which parentheses \verb|( )|
are used to permit the operator \verb|+| to be recognized, so that

\begin{indpar}\begin{verbatim}
##46:: straight 3.2; left; straight (y + 9.4)
\end{verbatim}\end{indpar}

is the cooked representation of the objects

\begin{indpar}\begin{verbatim}
##42:: 
    1: straight
    2: 3.2
##43:: 
    1: "+"
    2: y
    3: 9.4
    initiator: "("
    terminator: ")"
##44:: 
    1: straight
    2: ##43
##45::
    1: ##42
    2: left
    3: ##44
    .separator: ";"
\end{verbatim}\end{indpar}

Next is a third example in which
curly brackets (\verb|{ }|) are used with multiple
lines and indentation to represent code, so that

\begin{indpar}\begin{verbatim}
##138:: function (x,y)
        {
           if (x > y):
               return y
           else:
               return x
        }
\end{verbatim}\end{indpar}

is the cooked representation of

\begin{indpar}\begin{verbatim}
##130::
    1: x
    2: y
    .separator: ","
    .initiator: "("
    .terminator: ")"
##131::
    1: ">"
    2: x
    3: y
    .initiator: "("
    .terminator: ")"
##132::
    1: return
    2: y
    .initiator: ":"
##133::
    1: if
    2: ##131
    3: ##132
##134::
    1: return
    2: x
    .initiator: ":"
##135::
    1: else
    2: ##134
##136::
    1: ##133
    2: ##135
    .initiator: "{"
    .terminator: "}"
##137::
    1: function
    2: ##130
    2: ##136
\end{verbatim}\end{indpar}

Cooked representations may replace raw object names in the
description of other objects, as in

\begin{indpar}\begin{verbatim}
##291:: text: This is a sentence.
        outline: straight 3.2, left, straight (y + 9.4)
        min: function (x,y)
             {
                if (x > y):
                    return y
                else:
                    return x
             }
\end{verbatim}\end{indpar}

which is the cooked representation of the object

\begin{indpar}\begin{verbatim}
##291::
    text: ##93
    outline: ##45
    min: ##137
\end{verbatim}\end{indpar}

given the above examples.

One question left unanswered by the discussion so far is whether

\begin{indpar}\begin{verbatim}
##291:: text A: This is a sentence.
        text B: This is a sentence.
\end{verbatim}\end{indpar}

represents

\begin{indpar}\begin{verbatim}
##291::
    text A: ##93
    text B: ##93
\end{verbatim}\end{indpar}

or instead

\begin{indpar}\begin{verbatim}
##291::
    text A: ##93
    text B: ##999
\end{verbatim}\end{indpar}

where object \verb|##999| happens to have the same structure as
object \verb|##93|.  The default is to make both \verb|text A|
and \verb|text B| be the same object, \verb|##93|, and to make
that object `immutable', meaning that it cannot be changed.  The
rule is that unless otherwise indicated, only immutable objects
have the property that their cooked representations can replace
their raw object names in the cooked representations of other
objects.

The following are generalizations about certain attribute names.
First, the strictly positive integers \verb|1|, \verb|2|,
\verb|3|, \ldots are used to name the elements of a list.  Second,
some attribute names beginning with `\verb|.|' have special
meaning for object representation.  Examples above are
\verb|.separator|, \verb|.initiator|, and \verb|.terminator|.
Recall that words beginning with `\verb|.|' are reserved for
use by the MIN system, and should not be defined by MIN users.

In the following sections we describe cooked representations
precisely.

\subsection{Syntax}

A cooked data representation has the syntax given
in Figure~\figref{DATA-REPRESENTATION-SYNTAX} and is based on the
special lexemes given in Figure~\figref{SPECIAL-LEXEMES}.

\begin{boxedfigure}

\begin{center}

Ad Hoc Special Lexemes

\bigskip

\begin{tabular}{ll@{\hspace*{5em}}ll}
\ttkey{::} & object assignment &
\ttnbkey{>:} & object addition \\
\ttkey{:} & attribute assignment &
\ttkey{:} & code indicator \\
\end{tabular}

\bigskip

Brackets

\bigskip

\begin{tabular}{l@{~~~~~}l@{~~~~~}l}
brackets	& meaning	& parsing mode
\\[1ex]
\ttnbkey{(}~~~\ttnbkey{)}	& parentheses		& expression \\
\ttnbkey{[}~~~\ttnbkey{]}	& square brackets	& expression \\
\ttnbkey{\CurlyBra}~~~\ttnbkey{\CurlyKet}
				& curly brackets	& code \\
\ttnbkey{`}~~~\ttnbkey{'}	& single quotes		& text \\
\end{tabular}

\bigskip

Standard Operators

\bigskip

\begin{tabular}{rl@{\hspace*{2em}}l@{\hspace*{2em}}l}

priority & parser	& operators	& meaning 
\\[2ex]
-5000 & \ttkey{sentence} &  \ttnbkey{.} ~~~ \ttnbkey{!} ~~~ \ttnbkey{?}
	& sentence terminators \\
-4000 & \ttkey{subsentence} &  \ttnbkey{;} & subsentence separator \\
-3000 & \ttkey{phrase} &  \ttnbkey{,} & phrase separator \\
+0000 & \ttkey{assign} &  \ttnbkey{<-} ~~~ \ttkey{BECOMES} & assignment \\
+1000 & \ttkey{logical} &  \ttkey{AND} ~~~ \ttkey{OR} ~~~ \ttkey{NOT}
	& logical and, or, not \\
+2000 & \ttkey{compare} &  \ttnbkey{=} ~~~ \ttnbkey{/=} ~~~
        \ttnbkey{!=} 			& equal, not equal, ditto \\
      &		      & \ttnbkey{<} ~~~ \ttnbkey{<=} ~~~ \ttnbkey{=<}
      & less than, less than or \\
      &		      &		&
      equal, ditto \\
      &		      & \ttnbkey{>} ~~~ \ttnbkey{>=} ~~~ \ttnbkey{=>}
      & greater than, greater \\
      &		      &		&
      than or equal, ditto \\
+3000 & \ttkey{sum} &  \ttnbkey{+} ~~~ \ttnbkey{-}
	& addition, subtraction \\
+3100 & \ttkey{product} &  \ttnbkey{*} ~~~ \ttnbkey{/} 
	& multiplication, division \\
+9999 & \ttkey{objname} &  \ttnbkey{\#\#}
	& object name \\
\end{tabular}

\end{center}

\caption{Special Lexemes}
\label{SPECIAL-LEXEMES}
\end{boxedfigure}

\begin{boxedfigure}

\begin{center}
\begin{tabular}{l}
\key{data-representation} ::= {\em object-representation} \ldots
\\[1ex]
\key{object-representation} ::=
\\\hspace*{1in}
    \begin{tabular}[t]{@{}rl@{}}
        & {\em object-header} {\em attribute-representation} \ldots \\
    $|$ & {\em object-header} {\em expression}
	  \ttmkey{;;}{in {\em object-representation}}
    	  {\em attribute-representation} \ldots \\[1ex]
    \end{tabular}
\\[1ex]
\key{object-header}
    \begin{tabular}[t]{@{}rl@{}}
    ::= & {\em object-name} \ttmkey{::}{in {\em object-header}} \\
    $|$ & {\em object-name} \ttmnbkey{>:}{in {\em object-header}} \\
    $|$ & {\em object-name} \ttmnbkey{>>:}{in {\em object-header}} \\[1ex]
    \end{tabular}
\\[1ex]
\key{attribute-representation}
    \begin{tabular}[t]{@{}rl@{}}
    ::= & {\em raw-attribute-representation}
          (\pagref{RAW-ATTRIBUTE-REPRESENTATION}) \\
    $|$ & {\em cooked-single-attribute-representation} \\
    $|$ & {\em cooked-double-attribute-representation} \\[1ex]
    \end{tabular}
\\[1ex]
\key{cooked-single-attribute-representation} ::=
\\\hspace*{1in}
	{\em attribute-label} {\em attribute-label-terminator}
	{\em text}
	\ttmkey{;;}{in {\em cooked-single-attribute}!{\em -representation}}
\\[1ex]
\key{cooked-double-attribute-representation} ::=
\\\hspace*{1in}
	{\em attribute-label} {\em attribute-label-terminator}
\\\hspace*{1in}
	{\em double-attribute-value}
	{\em double-attribute-value}\ldots
\\\hspace*{1in}
	{\em attribute-label-initiator} {\em attribute-label}
	\ttmkey{;;}{in {\em raw-double-attribute-representation}}
\\[1ex]
\key{text} ::= {\em expression} in text mode
\\[1ex]
\key{expression} ::= {\em expression(-9999)}
\\[1ex]
\key{expression}{\em ($N$)} ::=
	\{ {\em expression($N+1$)} $|$ {\em operator($N$)} \}\ldots
\\[1ex]
\key{operator}{\em ($N$)} ::= operator of priority $N$:
			      see Figure~\figref{SPECIAL-LEXEMES}
\\[1ex]
\key{expression}{\em (10000)} ::= {\em atom} $|$ {\em bracketed-expression}
\\[1ex]
\key{bracketed-expression} \begin{tabular}[t]{@{}rl@{}}
			    ::= & \verb|(| {\em expression} \verb|)| \\
			    $|$ & \verb|[| {\em expression} \verb|]| \\
			    $|$ & \verb|{| {\em code-block} \verb|}| \\
			    $|$ & \verb|`| {\em text} \verb|'| \\[1ex]
			    \end{tabular}
\\[1ex]
\key{code-block} ::= {\em expression} in code mode

\end{tabular}
\end{center}%
\index{label!attribute}%
\index{name!object}

\caption{Data Representation Syntax}
\label{DATA-REPRESENTATION-SYNTAX}
\end{boxedfigure}

The difference between cooked data and raw data is the presence
of expressions.  An expression is parsed to produce an atom or
an object which is called the \key{parsed expression}.
If the expression is part of a
{\em cooked-attribute-representation}, the atom or object becomes
a value of the attribute.  For example, the cooked representation

\begin{indpar}\begin{verbatim}
##401::
    text A: This is a sentence.
    expression B: ( x + 8 )
\end{verbatim}\end{indpar}

is equivalent to the raw representation

\begin{indpar}\begin{verbatim}
##401::
    text A: ##402
    expression B: ##403
##402::
    1: this
    2: is
    3: a
    4: sentence
    .initiator: capital
    .terminator: "."
##403::
    1: "+"
    2: x
    3: 8
    .initiator: ")"
    .terminator: "("
\end{verbatim}\end{indpar}

If the expression immediately follows
an {\em object-header}, and is before any {\em attribute-representa\-tions}
in the {\em object-representation}, then if the parsed expression is
an object, the attributes of the object produced are copied to become
attributes of the object represented, and if the parsed expression is
an atom, the atom becomes the value of the represented object attribute
named `\ttmkey{1}{defined in cooked representation}'.  For example,
the cooked representation

\begin{indpar}\begin{verbatim}
##410:: This is a sentence.
        x: 5
\end{verbatim}\end{indpar}

is equivalent to the raw representation

\begin{indpar}\begin{verbatim}
##410::
    1: this
    2: is
    3: a
    4: sentence
    .initiator: capital
    .terminator: "."
    x: 5
\end{verbatim}\end{indpar}

and the cooked representation

\begin{indpar}\begin{verbatim}
##411:: hi
        x: 5
\end{verbatim}\end{indpar}

is equivalent to the raw representation

\begin{indpar}\begin{verbatim}
##411::
    1: hi
    x: 5
\end{verbatim}\end{indpar}

The precise algorithm used in parsing is described below in
\secref{PARSING}.


\subsection{Parsing Modes}
\label{PARSING-MODES}

There are three \skey{parsing mode}s: text, expression, and code.
The parsing mode of any lexeme is determined by the innermost
brackets containing the lexeme, according to Figure~\figref{SPECIAL-LEXEMES}.
Thus if there are no innermost brackets,
or if the innermost brackets are \verb|` '|, the parsing mode is `text'.
If the innermost brackets are \verb|( )| or \verb|[ ]|
the mode is `expression'.  If the innermost brackets are \verb|{ }|
the mode is `code'.  Thus it is as if \verb|` '| changes to text mode,
\verb|( )| and \verb|[ ]| to expression mode, and \verb|{ }| to code mode.

In \key{expression mode} all operators are recognized,
line feeds and indentation are treated like any other white-space, and
`\verb|:|' is not recognized as a code indicator.

\ikey{Text mode}{text mode} is the same as expression mode except that
only some operators are recognized.  Operators with zero or negative
priority are recognized.  Operators of positive priority are only recognized
within subexpressions of operators of zero priority.

Thus if the lexeme sequence `\verb|move 5 * y, x <- x + 1|' is parsed in
text mode, the \verb|+| is recognized as an operator because it is in a
subexpression of the zero priority \verb|<-| operator, while the
\verb|*| is not recognized as an operator
because it is not in such a subexpression.  To recognize the \verb|*| as
an operator, one should put \verb|5 * y| in parentheses, as in
`\verb|move ( 5 * y ), x <- x + 1|'.

In \key{code mode}\label{CODE-MODE} all operators are recognized,
while line feeds, indentation, and the code indicator `\verb|:|' are
used to sequence statements within the code block and to indicate
code subblocks, according to the following rules:

\begin{enumerate}
\item
The column of the first non-white-space character inside \verb|{}| brackets
defines the initial column of the \key{code block} enclosed by the
brackets.

\item No line in a code block may be indented by less than its initial column.
Any line indented by more than the initial column must be indented by at
least 2 columns more than the initial column.

\item A code block is organized as a sequence of statements such that
each non-white-space character in the code block initial column begins a new
statement, and statements so begun are the only statements in the code block.

\item
If a statement contains a `\verb|:|' \key{code indicator}\label{CODE-INDICATOR}
mark as the last
lexeme of a line, the first column of the next lexeme in the statement
begins a \key{code subblock} that includes the rest of the statement.  The
first column of that next lexeme is the initial column of the code
subblock.  The subblock obeys rules 2 and 3 just given for code blocks.

\item
A code block and its surrounding \verb|{ }| brackets represent a list of
zero or more statements with
list \verb|.initiator| \verb|"{"| and list \verb|.terminator| \verb|"}"|.

\item
A code subblock and its preceding code indicator `\verb|:|'
represents a list of zero or more statements with
list \verb|.initiator| \verb|":"| and no list \verb|.terminator|.

\end{enumerate}


\subsection{Parsing}
\label{PARSING}

Parsing a sequence of lexemes is done in four phases:

\begin{list}{}{}
\item[(1)] \ikey{Lexeme Scanning}{lexeme scanning}%
				 \index{scanning!lexeme}~~~~
The sequence of lexemes is represented as an object
that is a list whose elements (values of the attributes
named {\tt 1}, {\tt 2}, {\tt 3}, \ldots) are atoms that are lexemes
and objects representing bracketed sublists of lexemes.  Quoted
atoms are specially represented by a list with one element and the
{\tt .initiator} attribute value `{\tt quoted}'.

\item[(2)] \ikey{Operator Scanning}{operator scanning}%
				   \index{scanning!operator}~~~~
The object representation is scanned for operators
and modified by replacing subsequences of elements representing
subexpressions with implied parentheses with sublists, and by
adding {\tt .parser} attribute values to some of the list objects.

\item[(3)] \ikey{Parser Expansion}{parser expansion}%
				  \index{expansion!parser}~~~~
The object representation is evaluated by calling the parsers.

\item[(4)] \ikey{Dequoting}{dequoting}~~~~
Quoted atom representations are replaced by the atoms quoted.

\end{list}

\subsubsection{Lexeme Scanning}
\label{LEXEME-SCANNING}

The \key{lexeme scanning} algorithm inputs a sequence of lexemes
and outputs an object representing the sequence.  The object has
list elements which are attributes labeled {\tt 1}, {\tt 2}, \ldots.
The object may have an {\tt .initiator} and/or {\tt .terminator}
attribute.

The rules for computing the object from the lexeme sequence are:

\begin{enumerate}

\item The lexeme sequence is examined to identify matching brackets.
It is an error if there are unmatched brackets (bracket characters
in quoted strings do not count as brackets).

Then the first of the following rules which can be applied is applied,
unless no rule can be applied.

\begin{enumerate}

\item If the lexeme sequence begins and ends with matching brackets, these
are removed from the sequence, and become the {\tt .initiator} and
{\tt .terminator} attributes of the object being computed.

\item If the lexeme sequence begins with a `\verb|:|' code indicator mark
(\pagref{CODE-INDICATOR}), this
is removed from the sequence, and becomes the {\tt .initiator}
attribute of the object being computed.

\end{enumerate}

\item The remaining
lexeme sequence is then scanned from right to left to produce
the list elements of the object being computed (these become the
values of the attributes named {\tt 1}, {\tt 2}, {\tt 3}, \ldots).
At each stage of the
scan, lexemes are input and another list element is produced according
to the first of the following rules that can be applied.

\begin{enumerate}

\item If the next input is the beginning of a statement within a code
block (\pagref{CODE-MODE}),
the subsequence of lexemes that is the statement (including any
subblocks) becomes the input,
the lexical scanning algorithm is called recursively with this subsequence,
and the object output by the recursive call becomes the next list element.

\item If the next input is a bracket, the subsequence of lexemes beginning
with the bracket and ending with the matching bracket becomes the input,
the lexical scanning algorithm is called recursively with this subsequence,
and the object output by the recursive call becomes the next list element.

\item If the next input is a `\verb|:|' code indicator mark
(\pagref{CODE-INDICATOR}), the subsequence of lexemes consisting of
the code indicator mark and the following subblock becomes the input,
the lexical scanning algorithm is called recursively with this subsequence,
and the object output by the recursive call becomes the next list element.

\item If the next input is a quoted string, the next list element is an
object whose only attributes are an attribute named {\tt 1} whose value
is the quoted string atom,
and an {\tt .initiator} attribute whose value is the atom `{\tt quoted}'.

\item If the next input is a word or mark, the word or mark becomes the next
list element.

\end{enumerate}

\end{enumerate}

For example, given the input

\begin{center}
\verb|`The "+" in (x + 5 * y) is an operator.'|
\end{center}

the lexical scanning algorithm outputs object \verb|##1| in the
following:\label{LEXICAL-SCANNING-EXAMPLE-OUTPUT}

\begin{indpar}\begin{verbatim}
##1::
    1 : The
    2 : ##2
    3 : in
    4 : ##3
    5 : is
    6 : an
    7 : operator
    8 : .
    .initiator: "`"
    .terminator: "'"
##2::
    1: "+"
    .initiator: quoted
##3::
    1 : x
    2 : "+"
    3 : 5
    4 : "*"
    5 : y
    .initiator: "("
    .terminator: ")"
\end{verbatim}\end{indpar}

Recall that within objects there is no difference between
a quoted string representation of a mark and the mark, as both are
the same atom, but in textual representations of objects the quoted
string representation of marks that can be operators is used,
to keep the marks from being misread as operators.
Thus \verb|"+"| is used in the textual
object representation above to denote the 1 character atom \verb|+|,
and \verb|"*"|, \verb|"("|, \verb|"`"|, \verb|")"|, and \verb|"'"| 
also denote 1 character atoms.

\subsubsection{Operator Scanning}
\label{OPERATOR-SCANNING}

The \key{operator scanning} algorithm inputs an object output
by the lexical scanning algorithm and rearranges its contents according
to the operators it contains.
Specifically, subsequences of list elements between operators
are regrouped into sublists to represent subexpressions,
and {\tt .parser} attributes are added to
objects in preparation for expansion.

The operator scanning algorithm takes as input both an object and a parsing
mode.  The parsing mode is either text or expression (\pagref{PARSING-MODES}).
Code mode is handled elsewhere, during lexical scanning, and is treated as
equivalent to expression mode during operator scanning.  The output of
the algorithm is the input object which may be modified.
The operator scanning algorithm applies the following actions to the
object:

\begin{enumerate}

\item If the object has an 
an {\tt .initiator} attribute equal to `{\tt quoted}', the
scanning algorithm terminates without modifying the object.

\item
If the object has an {\tt .initiator} attribute,
the parsing mode argument to the algorithm is changed as follows:

\begin{center}
\begin{tabular}{lll}
\tt .initiator	& \tt .terminator	& new parsing mode \\[1ex]
\tt `		& \tt '			& text \\
\tt (		& \tt )			& expression \\
\tt [		& \tt ]			& expression \\
\LEFTBRACKET	& \RIGHTBRACKET		& expression \\
\tt :		& 			& expression \\
\end{tabular}
\end{center}
	
\item The object list elements are inspected to see if any are operators.
An operator is an atom equal to an operator that is recognized according
to the current parsing mode (see \pagref{PARSING-MODES},
Figure \figref{SPECIAL-LEXEMES} and \secref{OPERATOR-DECLARATION}).

If there are operators among the list elements, those with the lowest
priority L are selected.  Then any non-empty sequences of list elements
between selected operators are replaced by a single object listing the
replaced elements.  Then, the current object is given a {\tt .parser}
attribute equal to the parser associated with priority level L
(see Figure \figref{SPECIAL-LEXEMES} and \secref{OPERATOR-DECLARATION}).

Lastly, if the current parsing mode is text and L equals zero, the parsing
mode is changed to expression.

\item
The operator scanning algorithm is called recursively on each element
of the current object's list that is an object and not an atom.  In making
this recursive call, the current parsing mode is passed as an argument
to the recursive algorithm execution.


\end{enumerate}

Given the the lexical scanning output on
\pagref{LEXICAL-SCANNING-EXAMPLE-OUTPUT},
which was derived from the lexeme sequence,

\begin{center}
\verb|`The "+" in (x + 5 * y) is an operator.'|
\end{center}

operator scanning returns:

\begin{indpar}\begin{verbatim}
##1::
    1 : ##4
    2 : .
    .initiator: "`"
    .terminator: "'"
    .parser: sentence
##4::
    1 : The
    2 : ##2
    3 : in
    4 : ##3
    5 : is
    6 : an
    7 : operator
##2::
    1: "+"
    .initiator: quoted
##3::
    1 : ##5
    2 : "+"
    3 : ##6
    .initiator: "("
    .terminator: ")"
    .parser: sum
##5::
    1: x
##6::
    1 : ##7
    2 : "*"
    3 : ##8
    .parser: product
##7::
    1: 5
##8::
    1: y
\end{verbatim}\end{indpar}

Note that classical programming language prefix and postfix operators are
not supported by MIN.  Examples are given with descriptions of parsers
in the next section.\footnote{Classical prefix and postfix operators could be
supported by MIN, but seem to the author to be no more intuitive than
the MIN approach.  In particular, `{\tt x<y~AND NOT~y<z}' fails classically
but works in MIN.}

\subsubsection{Parser Expansion}
\label{PARSER-EXPANSION}

The \key{parser expansion} algorithm inputs an object output
by the operator scanning algorithm and outputs an
expanded object that replaces the input object.
The parser expansion algorithm goes through its input
looking for objects with a {\tt .parser} attribute, and for each
such object, calls its {\tt .parser} attribute value as a function with
the object as input and the function output as the expansion of the input
object.  The expansion algorithm works bottom up, from the innermost
expression outward, replacing subexpressions by their expansions.
It is thus an evaluation algorithm, with the
{\tt .parser} attribute values as the evaluators.

A {\tt .parser} attribute value is known as a \key{parser}, and is associated
with an operator priority level
(see Figure \figref{SPECIAL-LEXEMES} and \secref{OPERATOR-DECLARATION}).
The following are the standard parsers and their effects.

\begin{indpar}[1em]

\newcommand{\OP}[1]{\ttmkey{#1}{parsing}}
\newcommand{\NBOP}[1]{\ttmnbkey{#1}{parsing}}
\newcommand{\MAC}[2]{\hfill #1, \ttmkey{#2}{parsing macro}}

\bigskip

\ttmkey{sentence}{parser} \hfill
	\skey{terminator}s: \NBOP{.} ~ \NBOP{!} ~ \NBOP{?}

\begin{indpar}[0.5em]
Terminators may not be consecutive, may not begin an expression,
and must end an expression.  Thus each between-terminator subexpression
is non-empty and is followed by a terminator.
Each between-terminator subexpression is converted if necessary
to a list.  Then a \ttmkey{.terminator}{produced by parsing}
attribute is added to this list whose value is the terminator following the
subexpression.  If there is only one subexpression, the list it produces
is the parse result.  Otherwise the result is the list of all the subexpression
lists.
\end{indpar}

\bigskip


\ttmkey{subsentence}{parser} \hfill
	\key{subsentence separator}: \NBOP{;} \\
\ttmkey{phrase}{parser} \hfill
	\key{phrase separator}: \NBOP{,}

\begin{indpar}[0.5em]
Here `\verb|;|' or `\verb|,|' is the \key{separator}.  These have different
priority levels and different parsers, and so cannot appear together
in the same expression.

A list of all between-separator subexpressions is made, with empty
between-separator subexpressions being represented by objects with
no attributes.  There will be at least two subexpressions: one following
the last separator and one preceding the first separator.
The list of subexpressions is given a
\ttmkey{.separator}{produced by parsing}
attribute with value of the separator (\verb|";"| or \verb|","|),
and is returned as the result of the parse.
\end{indpar}

\bigskip

\ttmkey{assign}{parser} \hfill
	\skey{assign operator}s: \NBOP{<-} ~ \OP{BECOMES}

\begin{indpar}[0.5em]
Assign operators may not be consecutive and may not begin or end an expression.
The two different kinds of assignment operator, \verb|<-| and \verb|BECOMES|,
may not be mixed in the same expression.
The results rewritten using the two-argument assignment function \verb|<-|,
making multiple assignments from right to left.

\begin{indpar}[0.5em]
\verb|x <- y <- z + w| ~~~ $\Longrightarrow$ ~~~
\verb|<- x (<- y (+ x w))| \\
\verb|x BECOMES y BECOMES z + w| ~~~ $\Longrightarrow$ ~~~
\verb|<- x (<- y (+ x w))|
\end{indpar}
\end{indpar}

\bigskip

\ttmkey{logical}{parser} \hfill
	\skey{logical operator}s: \OP{AND} ~ \OP{OR} ~ \OP{NOT}

\begin{indpar}[0.5em]
One expression cannot have both {\tt AND} and {\tt OR} operators
outside explicitly bracketed subexpressions.  The {\tt NOT} operator
can only occur at the beginning of an expression or just after {\tt AND}
or {\tt OR}.  Subexpressions cannot be empty, except for the subexpressions
before a {\tt NOT}, which must always be empty.

If the only operator is {\tt NOT}, the expression is its own expansion.

Otherwise the expression is rewritten using a multi-argument {\tt AND}
or {\tt OR} function and a single argument {\tt NOT} function.

\begin{indpar}[0.5em]
\verb|NOT x AND NOT y AND z AND w| ~~~ $\Longrightarrow$ ~~~
\verb|AND (NOT x) (NOT y) z w| \\
\verb|NOT x = 8 OR NOT y < 9 OR z = 0| ~~~ $\Longrightarrow$
\\\hspace*{1in}
\verb|OR (NOT (x = 8)) (NOT (y < 9)) (z = 0)|
\end{indpar}

Note that {\tt NOT} is not a classical prefix operator, which would have
higher priority than \verb|=| or \verb|<|.  Also note that {\tt AND}
and {\tt OR} may not be used in the same logical expression without using
explicit parentheses.
\end{indpar}

\bigskip

\ttmkey{compare}{parser} \hfill
	\skey{compare operator}s:
	\NBOP{=} ~ \NBOP{<} ~ \NBOP{>} ~
	\NBOP{/=} ~ \NBOP{!=} ~
	\NBOP{=>} ~ \NBOP{>=} ~
	\NBOP{=<} ~ \NBOP{<=}

\begin{indpar}[0.5em]
Subexpressions cannot be empty, and a compare operator may not
begin or end an expression.
The expression is rewritten using a multi-argument {\tt AND}
function and two-argument compare functions.  Temporary variables
(\secref{TEMPORARY-VARIABLES})
are used to name intermediate expression values so as to
avoid recomputing arguments.

\begin{indpar}[0.5em]
\verb|x < y < z| ~~~ $\Longrightarrow$ ~~~
\verb|AND (< x (<- (.tmp 56) y)) (< (.tmp 56) z)| \\
\verb|x = y + z != w/2 <= x| ~~~ $\Longrightarrow$
\\\hspace*{1in}\begin{tabular}{@{}ll@{}}
	       \verb|AND| & \verb|(= x (<- (.tmp 57) (+ y z)))| \\
			  & \verb|(!= (.tmp 57) (<- (.tmp 58) (/w 2)))| \\
			  & \verb|(<= (.tmp 58) x)| \\
	       \end{tabular}
\end{indpar}

Note that sequences of comparison operators are treated as they are
in mathematics and not as they are in classical programming languages.
E.g., \verb|x<y<z| means \verb|x<y AND y<z| and \underline{not}
\verb|(x<y)<z|.
\end{indpar}

\bigskip

\ttmkey{sum}{parser} \hfill
	\skey{addition operator}s: \NBOP{+} ~ \NBOP{-}

\begin{indpar}[0.5em]
Addition operators may not be consecutive and may not end an expression.
The results rewritten using the multi-argument summation function \verb|+|
and the unary negation function \verb|-|.

\begin{indpar}[0.5em]
\verb|- x + y + z - w| ~~~ $\Longrightarrow$ ~~~
\verb|+ (- x) y z (- w)|
\end{indpar}
\end{indpar}

\bigskip

\ttmkey{product}{parser} \hfill
	\skey{multiplication operator}s: \NBOP{*} ~ \NBOP{/}

\begin{indpar}[0.5em]
The two different kinds of multiplication operators, \verb|*| and \verb|/|,
cannot be mixed in the same expression.
Multiplication operators may not be consecutive and may neither begin
nor end an expression.  The division operator \verb|/| must have exactly
two operands.
Between operator subexpressions are parsed, and the results rewritten using
the multi-argument multiplication function \verb|*| or the binary
division function \verb|/|.  E.g.:

\begin{indpar}[0.5em]
\verb|x * y * z| ~~~ $\Longrightarrow$ ~~~ \verb|* x y z| \\
\verb|x / y| ~~~ $\Longrightarrow$ ~~~ \verb|/ x y|

\end{indpar}
\end{indpar}


\end{indpar}

\subsubsection{Dequoting}
\label{DEQUOTING}

The \key{dequoting} algorithm inputs an object output
by the parser expansion algorithm and outputs a dequoted object.
If the object has an {\tt .initiator} attribute equal
to `{\tt quoted}', the single element of the object's list
(which will be an atom) is output.  Otherwise the dequoting algorithm is applied
to each element of the object's list that is itself an object (and not an
atom).



\section{Evaluation}

Consider the following description of a room:

\begin{indpar}\begin{verbatim}
##523::
    type: standard room
    outline: straight width, left, straight length, left,
             straight width, left,
             doored wall width,
             close
    width: 3.2
    length: 9.6
    doored wall: function ( length ) {
                     d = (length - door width) / 2
                     straight d
                     door (door width)
                     straight d }
    door : standard room door
    label: color green, under over, noun room
    over: ##598
\end{verbatim}\end{indpar}

The information in this description is not immediately useful for
producing output, such as a video picture of the room, or text naming
the room.  Instead, parts of the description must be evaluated to produce
the picture or the name.

The goal of an evaluation is to produce \key{derived information} which is
frequently stored in a derived attribute of an object.  A \key{derived
attribute} is just any attribute that stores derived information.  Derived
attributes are often also \skey{hidden attribute}s, which are just
attributes whose names begin with `\ttmkey{.}{hidden attribute}', and which
are not normally output when the object is saved.

For example, a very important derived attribute of any object is its
inheritance list, which is the value of the object's \verb|.ancestors|
attribute, and which is a list of other objects from which the current
object inherits attributes.  In our example, the \verb|type| attribute
names another object, `{\tt standard room}', from which the room object
inherits attributes.  This means that when an evaluation searches for the
name of an attribute in the room object, if that name is not found, the
evaluation will next search the `{\tt standard room}' object.

To compute the \verb|.ancestors| attribute an the \verb|type| object
is evaluated `in global context'.  This last implies that the only
names the evaluator can use are global variable names, and `{\tt standard
room}' in fact names a global variable whose value is the object
which is the sole immediate ancestor of our room.

The \verb|outline| attribute value is used to compute a derived attribute
named \verb|.video| which describes how the object is to be displayed
graphically.
The \verb|outline| attribute value is a program is written from the point
of view of a robot moving around the outline of the room building walls.
The robot follows a sequence of commands that are separated by commas.

In order to show how evaluation works, we will explain in detail some
of the evaluation of \verb|outline|.

Evaluation of the \verb|outline| attribute of object \verb|##523|
is triggered when the value
of the \verb|.video| attribute of this object is needed.  Computation
of the \verb|.video| attribute is be done by a function
named {\tt .make .video}, which in this case is defined by {\tt standard
room} which is an ancestor of \verb|##523|.
If this function did not exist, computation would done by the function
named {\tt .make}, which is globally defined, and which would attempt
to evaluate the {\tt video} attribute (without the `\verb|.|') of
\verb|##523| to produce this object's \verb|.video| attribute
(with the `\verb|.|').  In any case, computation of an attribute AA
can be done in a fashion tailored to what is being computed by defining a
`\verb|.make| AA' function in an object or one of its ancestors.

In our case the \verb|.make .video| function evaluates
the \verb|outline| attribute of \verb|##523| to produce
a value for the \verb|.video| attribute of \verb|##523|.
First the function establishes
a context, which is a list of objects that will be searched when an
attribute name is to be located.  The context for the evaluation of
an object's attribute value usually begins with the object and ends with
a special object called `\verb|.GLOBAL|'.  A global variable is by definition
just an attribute of the \verb|.GLOBAL| object.  An function evaluating
an {\tt outline} attribute is likely to include in the context an object
that defines functions named `{\tt straight}' and `{\tt left}' which will
evaluate parts of the {\tt outline} attribute.

Besides establishing a context, the \verb|.make| function establishes an
execution frame.  The execution frame may contain output
channels, which are places to put output.  In this case the value of
\verb|.video| becomes an output channel into which functions like
`{\tt straight}' will write new commands written in a video language
that can be processed by the video engine to produce a picture.

Since functions such as `{\tt straight}' and `{\tt left}' are written
from the point of view of a robot moving around the outline of a room,
the execution frame also contains the current location and direction
of the robot.

After establishing a context and frame, the evaluating function starts reading
the \verb|outline| attribute value which is to be evaluated.  In this
case the it finds a comma separated list of commands, and evaluates
each command separately.

The first command is `{\tt straight width}'.   When `{\tt straight}' is 
read, a search is made in the current context for an attribute with a name
beginning with the word `{\tt straight}'.  One will be found whose total
name is just the single word `{\tt straight}', and the value of that
attribute will be taken.  We have not shown that attribute, but it will
exist and have a value of the form

\begin{center}
\verb|function ( w ) { |\ldots\verb|}|
\end{center}

This value means the expression we are reading should have next, after the
word `{\tt straight}', a single argument.  An argument
is either a single lexeme or a bracketed string of lexemes.  The argument
must be read, and perhaps evaluated and replaced by its value, which becomes
the value of the \verb|w| argument of the `{\tt straight}' function.
Then this function is evaluated.

In our case `{\tt width}' follows `{\tt straight}'.  This is an unbracketed
single lexeme argument, and as such must be evaluated and replaced by
its value.  A search is made for an attribute whose name begins with
`{\tt width}'.  The `{\tt width}' attribute of \verb|##523| is found,
and as its value is a number,
\verb|3.2|, that is the final value of the `{\tt width}' argument.
Thus the command `{\tt straight width}' has become `{\tt straight 3.2}',
and this is then evaluated by executing the `{\tt straight}' function
with \verb|w| equal to \verb|3.2|.  The result will be some command
written into the {\tt .video} attribute of \verb|##523|.

The other commands in the {\tt outline} attribute are similar, except
for the `{\tt doored wall width}' command.  This invokes the `{\tt doored wall}'
function we have defined as an attribute of \verb|##523| (but it would
make more sense to define it as an attribute of the {\tt standard room}
object whose attributes are inherited by \verb|##523|).   Again
`{\tt width}' is the argument which evaluates to \verb|3.2| and becomes
the value of the `{\tt length}' argument in the `{\tt doored wall}'
function.  Evaluation of the `{\tt doored wall}' function is next,
and as a function,
its evaluation context begins with an object called the function frame
that has an attribute named `{\tt length}' whose value is the corresponding
argument value, in this case \verb|3.2|.

A special feature of the MIN language is are dependency lists which
can be used to figure out when to re-compute attribute values, such as
the \verb|.video| attribute of \verb|##523|.  Each attribute X can have
a dependency list which specifies all the other attributes Y, Z, \ldots
whose values depend upon the value of X.  When the value of X is changed,
the values of Y, Z, \ldots are marked `obsolete', which identifies them
as needing to be recomputed.  Thus in the current case the \verb|width|
attribute of \verb|##523| has a dependency list that includes the
\verb|.video| attribute of \verb|##523|.  Not all attributes will have
dependency lists: the \verb|.ancestors| attribute of an object would
have a very long dependency list, if it existed, that included all the
attributes ever computed from the object, and for this reason the
\verb|.ancestors| attribute has no dependency list, so changing it does
not mark other values obsolete.

We also need to explain
the meaning of the `{\tt label}' and `{\tt over}' attribute values
of \verb|##523|.  It is necessary to produce a name
for \verb|##523| in an arbitrary natural language, such as English or
French or Japanese, and this is what the `{\tt label}' attribute value does
when it is evaluated.  This value is also a sequence of commands that
provide descriptive information about the object which the natural language
engine can process to produce the desired text.  The `{\tt color green}'
command invokes the `{\tt color}' function with an argument that is
some object which is the value of the `{\tt green}' variable to
produce some internal notation that tells the natural language engine
that \verb|##523| has the color green.  Similarly `{\tt over under}'
invokes the `{\tt over}' function with an the `{\tt under}' argument that
just names an attribute of \verb|##523| with value \verb|##598|.  This
says that \verb|##523| is under \verb|##598|.  Lastly, `{\tt noun room}'
says that \verb|##523| is a room.

If the label of \verb|##598| is

\begin{center}
\verb|label: adjective main, noun dungeon|
\end{center}

then the English natural language engine might generate the text

\begin{center}
\verb|the green room under the main dungeon|
\end{center}

in order to specify \verb|##523|.

\subsection{The Evaluation Algorithm}

A CONTEXT is:

\begin{indpar}\begin{verbatim}
FRAME
EXPRESSION
NEXT ITEM INDEX         (of next ITEM in EXPRESSION)
STATE, and integer containing the following flags:
    CONDITION CODE:     CC TRUE, CC FALSE, CC NONE
                        (set by current statement)
    PREVIOUS CONDITION CODE:    CC TRUE, CC FALSE, CC NONE
                                (as set by previous statement)
    SKIP FLAG       (set to skip rest of current statement)
    REPEAT FLAG     (set to repeat current statement)
    RETURN FLAG     (set to return from current block)
    THROW FLAG      (set to throw from current block)
PREVIOUS CONTEXT    (contexts are like CONS cells)
\end{verbatim}\end{indpar}

A FRAME is:

\begin{indpar}\begin{verbatim}
... local variables are like object attributes ...
.object
.context
PREVIOUS FRAME           (frames are like CONS cells)
\end{verbatim}\end{indpar}

\begin{verbatim}

push ( CONTEXT ):
    C = copy of CONTEXT
    C PREVIOUS CONTEXT = CONTEXT
    return C

read ( CONTEXT ):
    if there is no next item in CONTEXT:
        return FAILURE
    V = CONTEXT EXPRESSION ( CONTEXT NEXT ITEM INDEX )
    CONTEXT NEXT ITEM INDEX = CONTEXT NEXT ITEM INDEX + 1
    return V

evaluate ( CONTEXT ):
    E = read ( CONTEXT )
    return evaluate ( E, CONTEXT )

evaluate ( EXPRESSION, CONTEXT ):
    if EXPRESSION is a single number atom:
        return EXPRESSION
    else if EXPRESSION is `...' bracketed list:
        return quoted ( EXPRESSION, CONTEXT, 1 )
    else if EXPRESSION is {...} bracketed list:
        return code ( EXPRESSION, CONTEXT )
    else:
        C = push CONTEXT
        C EXPRESSION = EXPRESSION
        C NEXT ITEM INDEX = 1
        V = scan ( C )
        CONTEXT STATE = C STATE
        discard C
        return V

quoted ( EXPRESSION, CONTEXT, DEPTH ):
    `Handles depth, e.g., `foo `fum [[x]] [y]'''
    `[[x]] is like LISP ,', - there is no equivalent of LISP ,,'
    RESULT = EXPRESSION with 0 list items
             (but same .initiator, etc.)
    foreach ITEM in EXPRESSION:
        if ITEM is not sublist:
            append ITEM to RESULT
            continue
        if ITEM is [...] bracketed:
            D = depth of ITEM [...] brackets
            if DEPTH > D:
                V = quote ( ITEM, CONTEXT, DEPTH - 1 )
                if CONTEXT STATE RETURN FLAG
                   or
                   CONTEXT STATE THROW FLAG
                    return V
                append V to end of RESULT
            else:
                V = evaluate ( ITEM, CONTEXT )
                if CONTEXT STATE RETURN FLAG
                   or
                   CONTEXT STATE THROW FLAG
                    return V
                concatenate V at end of RESULT
        else if ITEM is `...' quoted:
            V = quote ( ITEM, CONTEXT, DEPTH + 1 )
            if CONTEXT STATE RETURN FLAG
               or
               CONTEXT STATE THROW FLAG
                return V
            append V to end of RESULT
        else:
            V = quote ( ITEM, CONTEXT, DEPTH )
            if CONTEXT STATE RETURN FLAG
               or
               CONTEXT STATE THROW FLAG
                return V
            append V to end of RESULT
    return RESULT

code ( EXPRESSION, CONTEXT ):
    `EXPRESSION is {...} bracketed'
    `executing {...} code clears CC' 
    V = NULL
    CC = CC NONE
    foreach STATEMENT in EXPRESSION:
        CONTEXT STATE = REPEAT FLAG
        while CONTEXT STATE REPEAT FLAG:
            CONTEXT STATE REPEAT FLAG = 0
            CONTEXT STATE PREVIOUS CC = CC
            CONTEXT STATE CC = CC NONE
            V = evaluate ( STATEMENT, CONTEXT )
            CC = CONTEXT STATE CC
            CONTEXT STATE SKIP FLAG = 0
            if CONTEXT STATE RETURN FLAG
               or
               CONTEXT STATE THROW FLAG
                break
        CONTEXT STATE REPEAT FLAG = 0
        if CONTEXT STATE RETURN FLAG
           or
           CONTEXT STATE THROW FLAG
            break
    return V

scan ( CONTEXT ):
    V = scan ( CONTEXT, CONTEXT FRAME )
    if V != FAILURE: return V
    return scan ( CONTEXT, CONTEXT FRAME .object )

scan ( CONTEXT, OBJECT ):
   for each ANCESTOR of OBJECT:
       V = scan object ( CONTEXT, ANCESTOR )
       if V == FAILURE: break
   return V

scan object ( CONTEXT, OBJECT ):
   if CONTEXT STATE SKIP FLAG:
       return OBJECT
   if CONTEXT NEXT ITEM INDEX > 1,
      and OBJECT has an attribute named `.empty' with value V:
          return scan ( CONTEXT, V )
   if CONTEXT NEXT ITEM INDEX > length CONTEXT EXPRESSION:
        return OBJECT

   match names of OBJECT's attributes to next items in
         CONTEXT
   if several match:
      pick the longest match
      remove the matching items from CONTEXT
      let V = value of matched attribute
      if V is function definition:
           return execute ( CONTEXT, V, OBJECT )
      else:
           return scan ( CONTEXT, V )
   else:
       return FAILURE

execute ( CONTEXT, FUNCTION, CONTAINER ):
    make a new function frame F
    for each argument name N in FUNCTION:
        V = evaluate ( CONTEXT )
        if CONTEXT STATE RETURN FLAG
           or
           CONTEXT STATE THROW FLAG:
            discard F
            return V
        make an attribute of F with name N and
             value V
    F .context = CONTEXT
    F .object = CONTAINER
    C = push CONTEXT
    C FRAME = F
    V = code ( body of FUNCTION, C )
    C STATE RETURN FLAG = 0
    CONTEXT STATE = C STATE
    discard C,F
    return V

return builtin function ( V )
    .context state return flag = 1
    return V

try builtin function ()
    V = evaluate ( .context )
    if .context state throw flag:
        .context state throw flag = 0
        return evaluate ( .context )
    else:
        .context state skip flag = 1
        return V

\end{verbatim}



\subsection{Expressions and Subexpressions}

An \key{expression} is just a list of atoms and bracketed
subexpressions:


\subsection{Search}

\ikey{Search}{search} is part of the evaluation process.

Search is invoked with a \key{context}, which is just a list of
objects, and an expression, which is just a list of atoms.
The goal of search is to find an attribute whose name begins the expression.

\section{To Do}
Decapitalization\label{DECAPITALIZATION}

Operator Declaration\label{OPERATOR-DECLARATION}

Temporary Variables\label{TEMPORARY-VARIABLES}

Garbage Collection\label{GARBAGE-COLLECTION}


\bibliographystyle{plain}
\bibliography{min}

\printindex

\end{document}


