% Minimal Descriptive Programming Language Design
%
% File:         min-design.tex
% Author:       Bob Walton (walton@deas.harvard.edu)
% Date:		See \date below.
  
\documentclass[12pt]{article}

\usepackage{makeidx}
\usepackage{pictex}

\makeindex

\setlength{\oddsidemargin}{0in}
\setlength{\evensidemargin}{0in}
\setlength{\textwidth}{6.5in}
\raggedbottom

\setlength{\unitlength}{1in}

\pagestyle{headings}
\setlength{\parindent}{0.0in}
\setlength{\parskip}{1ex}

\setcounter{secnumdepth}{5}
\setcounter{tocdepth}{5}
\newcommand{\subsubsubsection}[1]{\paragraph[#1]{#1.}}
\newcommand{\subsubsubsubsection}[1]{\subparagraph[#1]{#1.}}

% Begin \tableofcontents surgery.

\newcount\AtCatcode
\AtCatcode=\catcode`@
\catcode `@=11	% @ is now a letter

\renewcommand{\contentsname}{}
\renewcommand\l@section{\@dottedtocline{1}{0.1em}{1.4em}}
\renewcommand\l@table{\@dottedtocline{1}{0.1em}{1.4em}}
\renewcommand\tableofcontents{%
    \begin{list}{}%
	     {\setlength{\itemsep}{0in}%
	      \setlength{\topsep}{0in}%
	      \setlength{\parsep}{1ex}%
	      \setlength{\labelwidth}{0in}%
	      \setlength{\baselineskip}{1.5ex}%
	      \setlength{\leftmargin}{1.0in}%
	      \setlength{\rightmargin}{1.0in}}%
    \item\@starttoc{toc}%
    \end{list}}
\renewcommand\listoftables{%
    \begin{list}{}%
	     {\setlength{\itemsep}{0in}%
	      \setlength{\topsep}{0in}%
	      \setlength{\parsep}{1ex}%
	      \setlength{\labelwidth}{0in}%
	      \setlength{\baselineskip}{1.5ex}%
	      \setlength{\leftmargin}{1.0in}%
	      \setlength{\rightmargin}{1.0in}%
	      }%
    \item\@starttoc{lot}%
    \end{list}}

\catcode `@=\AtCatcode	% @ is now restored

% End \tableofcontents surgery.

\newcommand{\CN}[2]%	Change Notice.
    {\hspace*{0in}\marginpar{\sloppy \raggedright \it \footnotesize
     $^{\mbox{#1}}$#2}}
    % Change notice.

\newcommand{\key}[1]{{\em #1}\index{#1}}
\newcommand{\mkey}[2]{{\em #1}\index{#1!#2}}
\newcommand{\skey}[2]{{\em #1#2}\index{#1}}
\newcommand{\ikey}[2]{{\em #1}\index{#2}}
\newcommand{\ttkey}[1]{{\tt #1}\index{#1@{\tt #1}}}
\newcommand{\ttmkey}[2]{{\tt #1}\index{#1@{\tt #1}!#2}}
\newcommand{\ttfkey}[2]{{\tt #1}\index{#1@{\tt #1}!for #2@for {\tt #2}}}
\newcommand{\ttakey}[2]{{\tt #1}\index{#2@{\tt #1}}}
\newcommand{\ttamkey}[3]{{\tt #1}\index{#2@{\tt #1}!#3}}
\newcommand{\ttindex}[1]{\index{#1@{\tt #1}}}
\newcommand{\ttmindex}[2]{\index{#1@{\tt #1}!#2}}
\newcommand{\emkey}[1]{{\em #1}\index{#1@{\em #1}}}
\newcommand{\emindex}[1]{\index{#1@{\em #1}}}

\newcommand{\secref}[1]{\ref{#1}$^{p\pageref{#1}}$}
\newcommand{\stepref}[1]{\ref{#1}$^{p\pageref{#1}}$}
\newcommand{\appref}[1]{\ref{#1}$^{p\pageref{#1}}$}
\newcommand{\pagref}[1]{p\pageref{#1}}

\newcommand{\EOL}{\penalty \exhyphenpenalty}

\newcount\TildeCatcode
\TildeCatcode=\catcode`\~
\catcode`~=12
\newcommand{\Tilde}{~}
\catcode`~=\TildeCatcode

\newcount\CircumflexCatcode
\CircumflexCatcode=\catcode`\^
\catcode`^=12
\newcommand{\Circumflex}{^}
\catcode`^=\CircumflexCatcode

\newcount\CurlyBraCatcode
\newcount\CurlyKetCatcode
\newcount\SquareBraCatcode
\newcount\SquareKetCatcode
\CurlyBraCatcode=\catcode`{
\CurlyKetCatcode=\catcode`}
\SquareBraCatcode=\catcode`[
\SquareKetCatcode=\catcode`]

\catcode`{=\SquareBraCatcode
\catcode`}=\SquareKetCatcode
\catcode`[=\CurlyBraCatcode
\catcode`]=\CurlyKetCatcode

\newcommand[\CurlyBra][{]
\newcommand[\CurlyKet][}]

\catcode`{=\CurlyBraCatcode
\catcode`}=\CurlyKetCatcode
\catcode`[=\SquareBraCatcode
\catcode`]=\SquareKetCatcode

\newcommand{\ttbrackets}{%
    \renewcommand{\{}{\CurlyBra}%
    \renewcommand{\}}{\CurlyKet}}

\newsavebox{\TILDEBOX}
\begin{lrbox}{\TILDEBOX}
\verb|~|
\end{lrbox}
\newcommand{\TILDE}{\usebox{\TILDEBOX}}

\newsavebox{\BACKSLASHBOX}
\begin{lrbox}{\BACKSLASHBOX}
\verb|\|
\end{lrbox}
\newcommand{\BACKSLASH}{\usebox{\BACKSLASHBOX}}

\newsavebox{\LEFTBRACKETBOX}
\begin{lrbox}{\LEFTBRACKETBOX}
\verb|{|
\end{lrbox}
\newcommand{\LEFTBRACKET}{\usebox{\LEFTBRACKETBOX}}

\newsavebox{\RIGHTBRACKETBOX}
\begin{lrbox}{\RIGHTBRACKETBOX}
\verb|}|
\end{lrbox}
\newcommand{\RIGHTBRACKET}{\usebox{\RIGHTBRACKETBOX}}

\newsavebox{\UNDERLINEBOX}
\begin{lrbox}{\UNDERLINEBOX}
\verb|_|
\end{lrbox}
\newcommand{\UNDERLINE}{\usebox{\UNDERLINEBOX}}

\newsavebox{\CIRCUMFLEXBOX}
\begin{lrbox}{\CIRCUMFLEXBOX}
\verb|^|
\end{lrbox}
\newcommand{\CIRCUMFLEX}{\usebox{\CIRCUMFLEXBOX}}

\newsavebox{\BARBOX}
\begin{lrbox}{\BARBOX}
\verb/|/
\end{lrbox}
\newcommand{\BAR}{\usebox{\BARBOX}}

\newsavebox{\LESSTHANBOX}
\begin{lrbox}{\LESSTHANBOX}
\verb/</
\end{lrbox}
\newcommand{\LESSTHAN}{\usebox{\LESSTHANBOX}}

\newsavebox{\GREATERTHANBOX}
\begin{lrbox}{\GREATERTHANBOX}
\verb/>/
\end{lrbox}
\newcommand{\GREATERTHAN}{\usebox{\GREATERTHANBOX}}

\newlength{\figurewidth}
\setlength{\figurewidth}{\textwidth}
\addtolength{\figurewidth}{-0.40in}

\newsavebox{\figurebox}

\newenvironment{boxedfigure}[1][!btp]%
	{\begin{figure*}[#1]
	 \begin{lrbox}{\figurebox}
	 \begin{minipage}{\figurewidth}

	 \vspace*{1ex}}%
	{
	 \vspace*{1ex}

	 \end{minipage}
	 \end{lrbox}
	 \begin{center}
	 \fbox{\hspace*{0.1in}\usebox{\figurebox}\hspace*{0.1in}}
	 \end{center}
	 \end{figure*}}

\newenvironment{indpar}[1][0.3in]%
	{\begin{list}{}%
		     {\setlength{\itemsep}{0in}%
		      \setlength{\topsep}{0in}%
		      \setlength{\parsep}{1ex}%
		      \setlength{\labelwidth}{#1}%
		      \setlength{\leftmargin}{#1}%
		      \addtolength{\leftmargin}{\labelsep}}%
	 \item}%
	{\end{list}}

\begin{document}
        
\title{Design\\[2ex]of the\\[2ex]
       Minimal\\Descriptive Programming\\Language\\[2ex]MIN\\[2ex]
       (Draft 1a)}

\author{Robert L. Walton}

\date{September 4, 2004}
 
\maketitle

\newpage
\begin{center}
\large \bf Table of Contents
\end{center}

\bigskip

\tableofcontents 

\newpage

\section{Introduction}

This document describes the internal design of MIN,
the Minimal Descriptive Programming Language.
This document is aimed at those who which to add C++ code
to a MIN implementation, or who wish to maintain an implementation.


\section{Data}

We first describe MIN data memory.  We then give two interfaces
to this memory: the protected interface, which can be used
by C++ code to access MIN data memory while maintaining the integrety
of that memory, and the unprotected interface, which provides
more efficient access to MIN data memory but requires the user to
follow certain protocols.

\subsection{Stubs and Bodies}

MIN data memory consists of regions that contain stubs and regions
that contain bodies.  A region is a continguous multi-page block
of memory.

\ikey{Stubs}{stub}
are small fixed size units of memory that cannot be relocated:
the usual stub size for MIN is 16 bytes.
Each object has a stub, and the address of the stub is in effect
the internal name of the object.  Some or all atoms, depending
on implementation, may have stubs.

A stub is divided into an 8 byte value and an 8 byte control.
The \mkey{value}{of stub} can hold an IEEE floating point number,
an 8 character string, or, as we will soon see, a pointer to a body.
It is also possible, though not common, for a value to hold any other
8 bytes of information.

The control holds a 1 byte type code and other information used,
for example, by the garbage collector.

A \key{body} is a variable sized relocatable block of memory
attached to a particular stub.  A stub may have one body attached to
it, in which case the stub value is a pointer to that body.
At almost any time the body may be moved and the stub value reset to
point at the new location of the body.  The body may be deallocated by
moving it to unimplemented memory.  Bodies are always some multiple
of 8 bytes long, and are allocated on 8 byte boundaries.

\subsection{Stub Control}

A stub contains an 8 byte value and an 8 byte \mkey{control}{of stub}.
If the control is viewed as a 64 bit integer, its high order byte
is the type code.  The high order bit of this, which is the high order
bit of the 64 bit control integer, is off if the stub is managed by
the garbage collector, and on otherwise.

If the stub is managed by the garbage collector, the control is used
exclusively by the garbage collector, except for the type code which
is shared between the garbage collector and the rest of the system.
In this case the stub is said to be `\key{collectable}'.  A typical
garbage collector organization of the control of a collectable stub is:

\begin{center}
\begin{tabular}{ll}
high order 8 bits:	& type code \\
next 8 bits:		& gc flags \\
low order 48 bits:	& chain pointer \\
\end{tabular}
\end{center}

The chain pointer is used to build lists of allocated stubs which
the garbage collector (gc) manages.

If a stub is non-collectable, its control can be organized in different
ways according to the type code value.  The standard way of organizing
the control is:

\begin{center}
\begin{tabular}{ll}
high order 8 bits:	& type code \\
next 8 bits:		& subtype code \\
low order 48 bits:	& chain pointer \\
\end{tabular}
\end{center}

The main use of non-collectable stubs is as auxilaries.
An `\key{auxilary}' is a non-collectable stub attached to an object.
When the object is garbage collected, the auxilary is freed.  Auxilaries
have two uses: first, as a means of adding memory to an object without
relocating the object, and seconds, as a means of adding additional
bodies to an object, since every body needs its own stub.

An example use of an auxilary to add memory to an object is in the
representation of double arrows.  The problem is that interally
a double arrow value must both point at the object that is the
arrow target and must also point at the label used by the target to
a double arrow value.  E.g, given

\begin{indpar}\begin{verbatim}
##1::
    fee: ##2 :fie

##2::
    fie: ##1 :fee
\end{verbatim}\end{indpar}

in which \verb|##1| and \verb|##2| are connected by a double
arrow that has the \verb|##1| attribute name \verb|fee| 
and the \verb|##2| attribute name \verb|fie|, then object \verb|##1|
must store as its \verb|fee| attribute value \underline{both}
the pointer to \verb|##2| that is the proper value of the attribute
\underline{and} the label \verb|fie| used by \verb|##2| to reference
the arrow in the other direction.  This is so that if the value of
the \verb|fee| attribute of \verb|##1| is changed, the \verb|fie|
attribute of \verb|##2| can be located and deleted.

The mechanism used to store the double pointer is:

\begin{indpar}\begin{verbatim}
 +---------------------------------------------------------+
 v                                                         |
##1::                                                      |
    fee: ---> auxilary 1:                                  |
 +------------- value = ##2                                |
 |              chain pointer ---> auxilary 2:             |
 |                                   value = fie           |
 |                                   chain pointer = NULL  |
 v                                                         |
##2::                                                      |
    fie: ---> auxilary 3:                                  |
                value = ##1 -------------------------------+
                chain pointer ---> auxilary 3:
                                     value = fee
                                     chain pointer = NULL
\end{verbatim}\end{indpar}

\bibliographystyle{plain}
\bibliography{min}

\printindex

\end{document}



